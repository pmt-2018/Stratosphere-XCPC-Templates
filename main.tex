
\documentclass[twoside]{article}
\usepackage[colorlinks,linkcolor=black]{hyperref}
\usepackage{xeCJK}
\usepackage{svg}
\usepackage{fancyhdr}
\usepackage{amsmath, amsthm}
\usepackage{listings,xcolor}
\usepackage{geometry}
\usepackage{fontspec}
\usepackage{graphicx}
\setsansfont{Monaco}
\setmonofont[Mapping={}]{Monaco}
\newcommand{\HRule}{\rule{\linewidth}{0.5mm}}
\geometry{left=2.5cm,right=2.5cm,top=2.5cm,bottom=2.5cm}
\definecolor{mygreen}{rgb}{0,0.6,0}
\definecolor{mygray}{rgb}{0.5,0.5,0.5}
\definecolor{mymauve}{rgb}{0.58,0,0.82}
\lstset{ 
  backgroundcolor=\color{white},        % choose the background color; you must add \usepackage{color} or \usepackage{xcolor}; should come as last argument
  basicstyle=\footnotesize\ttfamily,    % the size of the fonts that are used for the code
  breakatwhitespace=false,              % sets if automatic breaks should only happen at whitespace
  breaklines=true,                      % sets automatic line breaking
  captionpos=b,                         % sets the caption-position to bottom
  columns=fullflexible,
  commentstyle=\color{mygreen},         % comment style
  deletekeywords={...},                 % if you want to delete keywords from the given language
  escapeinside={\%*}{*)},               % if you want to add LaTeX within your code
  extendedchars=true,                   % lets you use non-ASCII characters; for 8-bits encodings only, does not work with UTF-8
  firstnumber=1,                     % start line enumeration with line 1000
  frame=single,	                        % adds a frame around the code
  keepspaces=true,                      % keeps spaces in text, useful for keeping indentation of code (possibly needs columns=flexible)
  keywordstyle=\color{blue},            % keyword style
  language=c++,                         % the language of the code
  lineskip=-0.2ex,
  morekeywords={*,...},                 % if you want to add more keywords to the set
  numbers=left,                         % where to put the line-numbers; possible values are (none, left, right)
  numbersep=5pt,                        % how far the line-numbers are from the code
  numberstyle=\tiny\color{mygray},      % the style that is used for the line-numbers
  rulecolor=\color{black},              % if not set, the frame-color may be changed on line-breaks within not-black text (e.g. comments (green here))
  rulesepcolor=\color{red!20!green!20!blue!20},
  showspaces=false,                     % show spaces everywhere adding particular underscores; it overrides 'showstringspaces'
  showstringspaces=false,               % underline spaces within strings only
  showtabs=true,                        % show tabs within strings adding particular underscores
  % stepnumber=2,                       % the step between two line-numbers. If it's 1, each line will be numbered
  stringstyle=\color{mymauve}\ttfamily, % string literal style
  tabsize=4,	                        % sets default tabsize to 2 spaces
  tab={\rule[-.2\baselineskip]{.4pt}{\baselineskip}\kern 1.5em}
}
\usepackage{indentfirst}
\setlength{\parindent}{2em}
\title{Stratosphere's XCPC Templates}
\author{平流层 Stratosphere}
\pagestyle{fancy}
\fancyhf{}
\fancyhead[C]{Stratosphere's XCPC Templates, 平流层 Stratosphere}
\begin{document}\small
\begin{titlepage}
\begin{center}
\vspace*{0.5cm}\includesvg[width=0.75\textwidth]{logo.svg} \\ [2cm]
\HRule \\ [1cm]
\textbf{\Huge{Stratosphere's XCPC Templates}} \\ [0.5cm]
\HRule \\ [4cm]
\textbf{\Huge{南京大学}} \\ [1cm]
\LARGE{平流层 Stratosphere}
\vfill
\Large{\today}
\end{center}
\clearpage
\end{titlepage}
\tableofcontents\clearpage
\pagestyle{fancy}
\lfoot{}
\cfoot{\thepage}\rfoot{}
\setcounter{section}{-1}
\setcounter{page}{1}
\clearpage\section{Header}
\clearpage\section{图论}
\subsection{欧拉回路}
\begin{lstlisting}
namespace Euler {
	bool directed;
	vector<pii> G[maxn];
	vector<int> ans;
	int vis[maxm];
	int dfs(int x) {
		vector<int> t;
		while (G[x].size()) {
			auto [to, id] = G[x].back();
			G[x].pop_back();
			if (!vis[abs(id)]) {
				vis[abs(id)] = 1, t.push_back(dfs(to)), ans.push_back(id);
			}
		}
		for (int i = 1; i < t.size(); i++) {
			if (t[i] != x) ans.clear();
		}
		return t.size() ? t[0] : x;
	}
	int n, m;
	pii e[maxm];
	int deg[maxn], vv[maxn];
	void clr() {
		for (int i = 1; i <= n; i++) G[i].clear(), deg[i] = vv[i] = 0;
		for (int i = 1; i <= m; i++) vis[i] = 0;
		ans.clear();
		n = m = 0;
	}
	void addedge(int x, int y) {
		ckmax(n, x), ckmax(n, y);
		e[++m] = {x, y};
		if (directed) {
			G[x].push_back({y, m});
			++deg[x], --deg[y], vv[x] = vv[y] = 1;
		} else {
			G[x].push_back({y, m});
			G[y].push_back({x, -m});
			++deg[x], ++deg[y], vv[x] = vv[y] = 1;
		}
	}
	using vi = vector<int>;
	pair<vi, vi> work() {
		if (!m) return clr(), pair<vi, vi>{{1}, {}};
		int S = 1;
		for (int i = 1; i <= n; i++)
			if (vv[i]) S = i;
		for (int i = 1; i <= n; i++)
			if (deg[i] > 0 && deg[i] % 2 == 1) S = i;
		dfs(S);
		if ((int)ans.size() != m) return clr(), pair<vi, vi>();
		reverse(ans.begin(), ans.end());
		vi ver, edge = ans;
		if (directed) {
			ver = {e[ans[0]].fi};
			for (auto t : ans) ver.push_back(e[t].se);
		} else {
			ver = {ans[0] > 0 ? e[ans[0]].fi : e[-ans[0]].se};
			for (auto t : ans) ver.push_back(t > 0 ? e[t].se : e[-t].fi);
		}
		clr();
		return {ver, edge};
	}
}  // namespace Euler\end{lstlisting}
\subsection{Tarjan-SCC}
\begin{lstlisting}
void tarjan(int u) {
	dfn[u] = low[u] = ++tim;
	in[u] = 1;
	st[++top] = u;
	for (int v : G[u]) {
		if (!dfn[v]) tarjan(v), ckmin(low[u], low[v]);
		else if (in[v]) ckmin(low[u], dfn[v]);
	}
	if (dfn[u] == low[u]) {
		++totc;
		int x;
		do { x = st[top--], in[x] = 0, bel[x] = totc; } while (x != u);
	}
}\end{lstlisting}
\subsection{点双}
\begin{lstlisting}
int T;  // assign = n
void tarjan(int u, int fa) {
	dfn[u] = low[u] = ++tim;
	stk[++top] = u;
	for (int v : G[u]) {
		if (v == fa) continue;
		if (!dfn[v]) dfs(v, u), ckmin(low[u], low[v]);
		else ckmin(low[u], dfn[v]);
	}
	if (fa && low[u] >= dfn[fa]) {
		int y;
		++T;
		do {
			y = stk[top--];
			G2[T].push_back(y), G2[y].push_back(T);
		} while (y != u);
		G2[T].push_back(fa), G2[fa].push_back(T);
	}
}\end{lstlisting}
\subsection{边双}
\begin{lstlisting}
void tarjan(int u, int f) {
	dfn[u] = low[u] = ++tim;
	st[++top] = u;
	for (int v : G[u]) {
		if (v == f) continue;
		if (!dfn[v]) tarjan(v, u), ckmin(low[u], low[v]);
		else ckmin(low[u], dfn[v]);
	}
	if (dfn[u] == low[u]) {
		++totc;
		int x;
		do { x = st[top--], in[x] = 0, bel[x] = totc; } while (x != u);
	}
}
\end{lstlisting}
\subsection{2-SAT}
\noindent 构造方案时可以通过变量在图中的拓扑序确定该变量的取值。\\
如果变量$x$的拓扑序在$\lnot x$之后,那么取$x$值为真。\\
因为Tarjan 算法求强连通分量时使用了栈,所以 Tarjan 求得的 SCC
编号相当于反拓扑序。
\begin{lstlisting}
for (int i = 1; i <= n; i++)
	if (bel[i << 1] == bel[i << 1 | 1]) return puts("IMPOSSIBLE"), 0;
puts("POSSIBLE");
for (int i = 1; i <= n; i++) printf("%d ", bel[i << 1] > bel[i << 1 | 1]);\end{lstlisting}
\subsection{最大流}
Dinic 算法
\begin{lstlisting}
namespace Dinic {
	int N, S, T;
	struct edge {
		int to, nxt, cap;
	} e[maxm << 1];
	int head[maxn], cur[maxn], tot = 1;
	int d[maxn];
	void addedge(int u, int v, int c) {
		e[++tot] = (edge){v, head[u], c}, head[u] = tot;
		e[++tot] = (edge){u, head[v], 0}, head[v] = tot;
	}
	bool bfs(int S, int T) {
		queue<int> q;
		for (int i = 1; i <= N; i++) d[i] = 0;
		d[S] = 1;
		q.push(S);
		while (!q.empty()) {
			int u = q.front();
			q.pop();
			for (int i = head[u]; i; i = e[i].nxt) {
				int v = e[i].to;
				if (e[i].cap && !d[v]) {
					d[v] = d[u] + 1, q.push(v);
					if (v == T) return true;
				}
			}
		}
		return false;
	}
	int dfs(int u, int f) {
		if (u == T) return f;
		int r = f;
		for (int& i = cur[u]; i && r; i = e[i].nxt) {
			int v = e[i].to;
			if (e[i].cap && d[v] == d[u] + 1) {
				int x = dfs(v, min(e[i].cap, r));
				if (!x) d[v] = 0;
				e[i].cap -= x, e[i ^ 1].cap += x;
				r -= x;
			}
		}
		return f - r;
	}
	ll work(int _N, int _S, int _T) {
		N = _N, S = _S, T = _T;
		ll ans = 0;
		while (bfs(S, T)) {
			for (int i = 1; i <= N; i++) cur[i] = head[i];
			ans += 1ll * dfs(S, INF);
		}
		return ans;
	}
}  // namespace Dinic
\end{lstlisting}
ISAP 算法
\begin{lstlisting}
namespace ISAP {
	int N, S, T;
	struct edge {
		int to, nxt, cap;
	} e[maxm << 1];
	int head[maxn], cur[maxn], gap[maxn], dis[maxn], tot = 1;
	void addedge(int u, int v, int w) {
		e[++tot] = {v, head[u], w}, head[u] = tot;
		e[++tot] = {u, head[v], 0}, head[v] = tot;
	}
	int ISAP(int u, int lim) {
		if (u == T) return lim;
		int res = 0;
		for (int& i = cur[u]; i; i = e[i].nxt) {
			int v = e[i].to;
			if (e[i].cap && dis[u] == dis[v] + 1) {
				ll det = ISAP(v, min(lim, e[i].cap));
				e[i].cap -= det, e[i ^ 1].cap += det;
				lim -= det, res += det;
				if (!lim) return res;
			}
		}
		cur[u] = head[u];
		if (!--gap[dis[u]]) dis[S] = N + 1;
		gap[++dis[u]]++;
		return res;
	}
	ll work(int _N, int _S, int _T) {
		S = _S, T = _T, N = _N;
		ll res = 0;
		while (dis[S] <= N) res += 1ll * ISAP(S, INF);
		return res;
	}
}  // namespace ISAP
\end{lstlisting}
HLPP 算法
\begin{lstlisting}
namespace HLPP {  // by ProjectEMmm
	int N, S, T;
	struct edge {
		int to, nxt, cap;
	} e[maxm << 1];
	int head[maxn], tot = 1;

	int d[maxn], num[maxn];
	stack<int> lib[maxn];
	ll ex[maxn];
	int level = 0;
	void addedge(int u, int v, int c) {
		e[++tot] = {v, head[u], c}, head[u] = tot;
		e[++tot] = {u, head[v], 0}, head[v] = tot;
	}
	int Push(int u) {
		bool init = (u == S);
		for (int i = head[u]; i; i = e[i].nxt) {
			const int &v = e[i].to, &c = e[i].cap;
			if (!c || init == false && d[u] != d[v] + 1) continue;
			ll k = init ? c : min((ll)c, ex[u]);
			if (v != S && v != T && !ex[v] && d[v] < INF)
				lib[d[v]].push(v), level = max(level, d[v]);
			ex[u] -= k, ex[v] += k, e[i].cap -= k, e[i ^ 1].cap += k;
			if (!ex[u]) return 0;
		}
		return 1;
	}
	void Relabel(int x) {
		d[x] = INF;
		for (int i = head[x]; i; i = e[i].nxt)
			if (e[i].cap) d[x] = min(d[x], d[e[i].to]);
		if (++d[x] < N) {
			lib[d[x]].push(x);
			level = max(level, d[x]);
			++num[d[x]];
		}
	}
	bool BFS() {
		for (int i = 1; i <= N; ++i) {
			d[i] = INF;
			num[i] = 0;
		}
		queue<int> q;
		q.push(T), d[T] = 0;
		while (!q.empty()) {
			int u = q.front();
			q.pop();
			num[d[u]]++;
			for (int i = head[u]; i; i = e[i].nxt) {
				const int& v = e[i].to;
				if (e[i ^ 1].cap && d[v] > d[u] + 1) d[v] = d[u] + 1, q.push(v);
			}
		}
		return d[S] != INF;
	}
	int Select() {
		while (lib[level].size() == 0 && level > -1) level--;
		return level == -1 ? 0 : lib[level].top();
	}
	ll work(int _N, int _S, int _T) {
		N = _N, S = _S, T = _T;
		if (!BFS()) return 0;
		d[S] = N;
		Push(S);
		int x;
		while (x = Select()) {
			lib[level].pop();
			if (!Push(x)) continue;
			if (!--num[d[x]])
				for (int i = 1; i <= N; ++i)
					if (i != S && i != T && d[i] > d[x] && d[i] < N + 1)
						d[i] = N + 1;
			Relabel(x);
		}
		return ex[T];
	}
}  // namespace HLPP\end{lstlisting}
\subsection{最小费用最大流}
\begin{lstlisting}
namespace MCMF {
	using pr = pair<ll, int>;
	int N, S, T;
	struct edge {
		int to, nxt, cap, w;
	} e[maxm << 1];
	int head[maxn], tot = 1;
	void addedge(int x, int y, int cap, int w) {
		e[++tot] = {y, head[x], cap, w}, head[x] = tot;
		e[++tot] = {x, head[y], 0, -w}, head[y] = tot;
	}
	ll d[maxn], dis[maxn];
	int vis[maxn], fr[maxn];
	bool spfa() {
		queue<int> Q;
		fill(d + 1, d + N + 1, 1e18);  // CHECK
		for (d[S] = 0, Q.push(S); !Q.empty();) {
			int x = Q.front();
			Q.pop();
			vis[x] = 0;
			for (int i = head[x]; i; i = e[i].nxt)
				if (e[i].cap && d[e[i].to] > d[x] + e[i].w) {
					d[e[i].to] = d[x] + e[i].w;
					fr[e[i].to] = i;
					if (!vis[e[i].to]) vis[e[i].to] = 1, Q.push(e[i].to);
				}
		}
		return d[T] < 1e17;  // 如果只是最小费用流,当d < 0继续增广
	}
	bool dijkstra() {  // 正常题目不需要 dijk
		priority_queue<pr, vector<pr>, greater<pr>> Q;
		for (int i = 1; i <= N; ++i)
			dis[i] = d[i], d[i] = 1e18, vis[i] = fr[i] = 0;  // CHECK
		Q.emplace(d[S] = 0, S);
		while (!Q.empty()) {
			int x = Q.top().second;
			Q.pop();
			if (vis[x]) continue;
			vis[x] = 1;
			for (int i = head[x]; i; i = e[i].nxt) {
				const ll v = e[i].w + dis[x] - dis[e[i].to];
				if (e[i].cap && d[e[i].to] > d[x] + v) {
					fr[e[i].to] = i;
					Q.emplace(d[e[i].to] = d[x] + v, e[i].to);
				}
			}
		}
		for (int i = 1; i <= N; ++i) d[i] += dis[i];  // CHECK
		return d[T] < 1e17;
	}
	std::pair<ll, ll> work(int _N, int _S, int _T) {
		N = _N, S = _S, T = _T;
		spfa();  // 如果初始有负权且要 dijk
		ll f = 0, c = 0;
		for (; dijkstra();) {  // 正常可以用 spfa
			ll fl = 1e18;
			for (int i = fr[T]; i; i = fr[e[i ^ 1].to])
				fl = min((ll)e[i].cap, fl);
			for (int i = fr[T]; i; i = fr[e[i ^ 1].to])
				e[i].cap -= fl, e[i ^ 1].cap += fl;
			f += fl, c += fl * d[T];
		}
		return make_pair(f, c);
	}
}  // namespace MCMF\end{lstlisting}
\subsection{匹配}
\subsubsection{二分图最大匹配-Hungary}
\begin{lstlisting}
// 匈牙利,左到右单向边, O (M |match|)
int vis[maxn], match[maxn];
bool dfs(int u) {
	for (int v : G[u]) {
		if (vis[v]) continue;
		vis[v] = 1;
		if (!match[v] || dfs(match[v])) return match[v] = u, 1;
	}
	return 0;
}
int work() {
	for (int i = 1; i <= nl; i++)
		if (dfs(i)) fill(vis + 1, vis + nr + 1, 0);
}
// 匈牙利,左到右单向边,bitset, O (n^2|match|/w)
bitset<N> G[N], unvis;
int match[N];
bool dfs(int u) {
	for (auto s = G[u];;) {
		s &= unvis;
		int v = s._Find_first();
		if (v == N) return 0;
		unvis.reset(v);
		if (!match[v] || dfs(match[v])) return match[v] = u, 1;
	}
	return 0;
}
int work() {
	unvis.set();
	for (int i = 1; i <= nl; i++)
		if (dfs(i)) unvis.set();
}\end{lstlisting}
\subsubsection{二分图最大匹配-HK}
\begin{lstlisting}
// HK, 左到右单向边, O(M \sqrt{|match|})
int nl, nr, m;
vi G[maxn];
int L[maxn], R[maxn], vis[maxn], matchl[maxn], matchr[maxn];
queue<int> Q;
bool bfs() {
	for (int i = 1; i <= nl; i++) L[i] = 0;
	for (int i = 1; i <= nr; i++) R[i] = 0;

	for (int i = 1; i <= nl; i++)
		if (!matchl[i]) L[i] = 1, Q.push(i);
	int succ = 0;
	while (!Q.empty()) {
		int u = Q.front();
		Q.pop();
		for (int v : G[u]) {
			if (R[v]) continue;
			R[v] = L[u] + 1;
			if (matchr[v]) {
				L[matchr[v]] = R[v] + 1, Q.push(matchr[v]);
			} else succ = 1;
		}
	}
	return succ;
}
bool dfs(int u) {
	for (int v : G[u])
		if (R[v] == L[u] + 1 && !vis[v]) {
			vis[v] = 1;
			if (!matchr[v] || dfs(matchr[v]))
				return matchl[u] = v, matchr[v] = u, 1;
		}
	return 0;
}
void HK() {
	while (bfs()) {
		for (int i = 1; i <= nr; i++) vis[i] = 0;
		for (int i = 1; i <= nl; ++i)
			if (!matchl[i]) dfs(i);
	}
	return;
}
\end{lstlisting}
\subsubsection{二分图最大权匹配-KM}
\begin{lstlisting}
// KM 二分图最大权匹配 复杂度O(n^3)
namespace KM {
	int nl, nr;
	ll e[maxn][maxn], lw[maxn], rw[maxn], mnw[maxn];
	int lpr[maxn], rpr[maxn], vis[maxn], fa[maxn];
	void addedge(int x, int y, ll w) { ckmax(e[x][y], w), ckmax(lw[x], w); }
	void work(int x) {
		int xx = x;
		for (int i = 1; i <= nr; i++) vis[i] = 0, mnw[i] = 1e18;
		while (true) {
			for (int i = 1; i <= nr; i++)
				if (!vis[i] && mnw[i] >= lw[x] + rw[i] - e[x][i])
					ckmin(mnw[i], lw[x] + rw[i] - e[x][i]), fa[i] = x;
			ll mn = 1e18;
			int y = -1;
			for (int i = 1; i <= nr; i++)
				if (!vis[i] && mn >= mnw[i]) ckmin(mn, mnw[i]), y = i;
			lw[xx] -= mn;
			for (int i = 1; i <= nr; i++)
				if (vis[i]) rw[i] += mn, lw[rpr[i]] -= mn;
				else mnw[i] -= mn;
			if (rpr[y]) x = rpr[y], vis[y] = 1;
			else {
				while (y) rpr[y] = fa[y], swap(y, lpr[fa[y]]);
				return;
			}
		}
	}
	void init(int _nl, int _nr) {
		nl = _nl, nr = _nr;
		if (nl > nr) nr = nl;
		for (int i = 1; i <= nl; i++) lw[i] = -1e18;
		for (int i = 1; i <= nl; i++)
			for (int j = 1; j <= nr; j++) e[i][j] = 0;  // or -1e18
	}
	ll work() {
		for (int i = 1; i <= nl; i++) work(i);
		ll tot = 0;
		for (int i = 1; i <= nl; i++) tot += e[i][lpr[i]];
		return tot;
	}
}  // namespace KM\end{lstlisting}
\subsubsection{一般图最大匹配-带花树}
\begin{lstlisting}
namespace blossom {
	vector<int> G[maxn];
	int f[maxn];
	int n, match[maxn];
	int getfa(int x) { return f[x] == x ? x : f[x] = getfa(f[x]); }
	void addedge(int x, int y) { G[x].push_back(y), G[y].push_back(x); }
	int pre[maxn], mk[maxn];
	int vis[maxn], T;
	queue<int> q;
	int LCA(int x, int y) {
		T++;
		for (;; x = pre[match[x]], swap(x, y))
			if (vis[x = getfa(x)] == T) return x;
			else vis[x] = x ? T : 0;
	}
	void flower(int x, int y, int z) {
		while (getfa(x) != z) {
			pre[x] = y;
			y = match[x];
			f[x] = f[y] = z;
			x = pre[y];
			if (mk[y] == 2) q.push(y), mk[y] = 1;
		}
	}
	void aug(int s) {
		for (int i = 1; i <= n; i++) pre[i] = mk[i] = vis[i] = 0, f[i] = i;
		q = {};
		mk[s] = 1;
		q.push(s);
		while (q.size()) {
			int x = q.front();
			q.pop();
			for (int v : G[x]) {
				int y = v, z;
				if (mk[y] == 2) continue;
				if (mk[y] == 1) z = LCA(x, y), flower(x, y, z), flower(y, x, z);
				else if (!match[y]) {
					for (pre[y] = x; y;)
						x = pre[y], match[y] = x, swap(y, match[x]);
					return;
				} else
					pre[y] = x, mk[y] = 2, q.push(match[y]), mk[match[y]] = 1;
			}
		}
	}
	int work() {
		for (int i = 1; i <= n; i++)
			if (!match[i]) aug(i);
		int res = 0;
		for (int i = 1; i <= n; i++) res += match[i] > i;
		return res;
	}
}  // namespace blossom\end{lstlisting}
\subsubsection{一般图最大权匹配}
待补充\subsection{流和匹配的建模技巧}
\subsubsection{二分图相关}
\begin{itemize}
    \item 二分图最小点覆盖:等于最大匹配$|match|$。从每一个非匹配点出发,沿着非匹配边正向进行遍历,沿着匹配边反向进行遍历到的点进行标记。选取左部点中没有被标记过的点,右部点中被标记过的点,则这些点可以形成该二分图的最小点覆盖。
    \item 二分图最大独立集:等于$n-|match|$,考虑最小点覆盖给所有边都至少有一边有点,取反后必然为最大独立集。
    \item 二分图最小边覆盖:等于$n-|match|$,考虑最坏情况每个顶点都要一条边,一个匹配能减小1的贡献。
    \item 最大团:等于补图的最大独立集。
    \item 最小路径覆盖:对于每条有向边$(u,v)$,拆成$u\to v+n$,$u$为进入$u$,$v+n$为从v离开,则答案为$n-|match|$。
    \item Hall Theorem: 对于左部顶点集$X$,$\forall S\subseteq X, |N(S)|\ge |S|\iff$ 存在完美匹配。
\end{itemize}

\subsubsection{网络流相关}

\begin{itemize}
    \item 二分图最大权独立集:考虑连边$(S, x, w_x)$,原图边$(x,y,\infty)$,$(y,T,w_y)$,变为最小割。
    \item 最大权闭合子图:正权$w_u$连$(S,u,w_u)$,负权$w_v$连$(v,T,-w_v)$,原图边连$\infty$。此时最小割之后源点$S$能到达的点即为最大权闭合子图,答案即为正权和$-\text{mincut}$。
    \item 无源汇上下界可行流:建源汇$S,T$,$l(u,v),r(u,v)$分别为流量上下界。记$d(i) = \sum l(u,i) - \sum l(i,v)$。
    \begin{itemize}
        \item 原边$(u,v)$连$(u,v,r(u,v)-l(u,v))$。
        \item 对于每个点$u$,若$d_u>0$,连$(S, u, d_u)$。
        \item 若$d_u<0$,连$(u, T, -d_u)$。
    \end{itemize}
    若$S$的出边全部流满则存在解。
    \item 有源汇上下界可行流:原图源汇连边$(T\to S,(0,\infty))$,则转化为无源汇。
    \item 有源汇上下界最大流:从 $T$ 到 $S$ 连一条下界为 $0$,上界为 $+\infty$ 的边,转化为无源汇网络。按照无源汇上下界可行流的做法求一次无源汇上下界超级源$SS$到超级汇$TT$的最大流。
    删去所有附加边,在上一步的\textbf{残量网络}基础上,求一次 $S$ 到 $T$ 的最大流。两者之和即为答案。
    \item 有源汇上下界最小流:从 $T$ 到 $S$ 连一条下界为 $0$,上界为 $+\infty$ 的边,转化为无源汇网络。按照无源汇上下界可行流的做法求一次无源汇上下界超级源$SS$到超级汇$TT$的最大流。
    删去所有附加边,在上一步的\textbf{残量网络}基础上,求一次 $T$ 到 $S$ 的最大流。两者之差即为答案。
    \item 最小费用可行流:同有源汇上下界可行流,在超级源汇跑最小费用最大流,答案为费用+下界流量的费用。
    \item 平面图最小割 = 对偶图最短路
\end{itemize}
\subsection{最短路相关}
\subsubsection{差分约束}
$x_i$ 向 $x_j$ 连一条权值为 $c$ 的有向边表示 $x_j - x_i \le c$。

用 BF 判断是否存在负环,存在即无解。\subsubsection{最小环}
记原图中 $u,v$ 之间边的边权为 $val\left(u,v\right)$。

我们注意到 Floyd 算法有一个性质:在最外层循环到点 $k$ 时(尚未开始第 $k$ 次循环),最短路数组 $dis$ 中,$dis_{u,v}$ 表示的是从 $u$ 到 $v$ 且仅经过编号在 $\left[1, k\right)$ 区间中的点的最短路。

由最小环的定义可知其至少有三个顶点,设其中编号最大的顶点为 $w$,环上与 $w$ 相邻两侧的两个点为 $u,v$,则在最外层循环枚举到 $k=w$ 时,该环的长度即为 $dis_{u,v}+val\left(v,w\right)+val\left(w,u\right)$。

故在循环时对于每个 $k$ 枚举满足 $i<k,j<k$ 的 $(i,j)$,更新答案即可。\subsubsection{Steiner树}
状态设计:$dp(i,S)$ 以 $i$ 为根,树中关键点集合为 $S$ 的最小值。

\begin{enumerate}
    \item[1.] 树根度数不为 $1$ ,考虑拆分成两个子集 $T,S-T$:
    $$ dp(i,S) \gets dp(i,S-T)+dp(i,T) $$
    \item[2.] 树根度数为 $1$ :
    $$ dp(i,S) \gets dp(j,S)+w(i,j) $$

    相当于超级源到每个顶点距离为$dp(i,S)$,求到每个顶点的最短路,dij即可。
\end{enumerate}
\subsection{三四元环计数}
\begin{lstlisting}
static int id[maxn], rnk[maxn];
for (int i = 1; i <= n; i++) id[i] = i;
sort(id + 1, id + n + 1, [](int x, int y) {
	return pii{deg[x], x} < pii{deg[y], y};
});
for (int i = 1; i <= n; i++) rnk[id[i]] = i;
for (int i = 1; i <= n; i++)
	for (int v : G[i])
		if (rnk[v] > rnk[i]) G2[i].push_back(v);
int ans3 = 0;  // 3-cycle
for (int i = 1; i <= n; i++) {
	static int vis[maxn];
	for (int v : G2[i]) vis[v] = 1;
	for (int v1 : G2[i])
		for (int v2 : G2[v1])
			if (vis[v2]) ++ans3;  // (i,v1,v2)
	for (int v : G2[i]) vis[v] = 0;
}
ll ans4 = 0;  // 4-cycle
for (int i = 1; i <= n; i++) {
	static int vis[maxn];
	for (int v1 : G[i])
		for (int v2 : G2[v1])
			if (rnk[v2] > rnk[i]) ans4 += vis[v2], vis[v2]++;
	for (int v1 : G[i])
		for (int v2 : G2[v1]) vis[v2] = 0;
}\end{lstlisting}
\subsection{支配树}
\begin{lstlisting}
namespace Dom_DAG {
	int idom[maxn];
	vector<int> G[maxn], ANS[maxn];  // ANS: final tree
	int deg[maxn];
	int fa[maxn][25], dep[maxn];
	int lca(int x, int y) {
		if (dep[x] < dep[y]) swap(x, y);
		for (int i = 20; i >= 0; i--)
			if (fa[x][i] && dep[fa[x][i]] >= dep[y]) x = fa[x][i];
		if (x == y) return x;
		for (int i = 20; i >= 0; i--)
			if (fa[x][i] != fa[y][i]) x = fa[x][i], y = fa[y][i];
		return fa[x][0];
	}
	void work() {
		queue<int> q;
		q.push(1);
		while (!q.empty()) {
			int x = q.front();
			q.pop();
			ANS[idom[x]].push_back(x);
			fa[x][0] = idom[x];
			dep[x] = dep[idom[x]] + 1;
			for (int i = 1; i <= 20; i++) fa[x][i] = fa[fa[x][i - 1]][i - 1];
			for (int v : G[x]) {
				--deg[v];
				if (!deg[v]) q.push(v);
				if (!idom[v]) idom[v] = x;
				else idom[v] = lca(idom[v], x);
			}
		}
	}
}  // namespace Dom_DAG
namespace Dom {
	vector<int> G[maxn], rG[maxn];
	int dfn[maxn], id[maxn], anc[maxn], cnt;
	void dfs(int x) {
		id[dfn[x] = ++cnt] = x;
		for (int v : G[x])
			if (!dfn[v]) {
				Dom_DAG::G[x].push_back(v);
				Dom_DAG::deg[v]++;
				anc[v] = x;
				dfs(v);
			}
	}
	int fa[maxn], mn[maxn];
	int find(int x) {
		if (x == fa[x]) return x;
		int tmp = fa[x];
		fa[x] = find(fa[x]);
		ckmin(mn[x], mn[tmp]);
		return fa[x];
	}
	int semi[maxn];
	void work() {
		dfs(1);
		for (int i = 1; i <= n; i++) fa[i] = i, mn[i] = 1e9, semi[i] = i;
		for (int w = n; w >= 2; w--) {
			int x = id[w];
			int cur = 1e9;
			if (w > cnt) continue;
			for (int v : rG[x]) {
				if (!dfn[v]) continue;
				if (dfn[v] < dfn[x]) ckmin(cur, dfn[v]);
				else find(v), ckmin(cur, mn[v]);
			}
			semi[x] = id[cur];
			mn[x] = cur;
			fa[x] = anc[x];
			Dom_DAG::G[semi[x]].push_back(x);
			Dom_DAG::deg[x]++;
		}
	}
	void addedge(int x, int y) { G[x].push_back(y), rG[y].push_back(x); }
}  // namespace Dom\end{lstlisting}
\subsection{图论计数}
\subsubsection{Prufer序列}
有标号无根树和其prufer编码一一对应, 一颗 $n$ 个点的树, 其prufer编码长度为 $n - 2$, 且度数为 $d_i$ 的点在prufer编码中出现 $di - 1$ 次.

由树得到序列: 总共需要 $n - 2$ 步, 第 $i$ 步在当前的树中寻找具有最小标号的叶子节点, 将与其相连的点的标
号设为Prufer序列的第 $i$ 个元素 $p_i$ , 并将此叶子节点从树中删除, 直到最后得到一个长度为 $n - 2$ 的Prufer 序
列和一个只有两个节点的树.

由序列得到树: 先将所有点的度赋初值为 $1$, 然后加上它的编号在Prufer序列中出现的次数, 得到每个点的度;
执行 $n - 2$ 步, 第 $i$ 步选取具有最小标号的度为 $1$ 的点 $u$ 与 $v = p_i$ 相连, 得到树中的一条边, 并将 $u$ 和 $v$ 的度
减一. 最后再把剩下的两个度为 $1$ 的点连边, 加入到树中.

推论:

\begin{itemize}
    \item $n$ 个点完全图, 要求每个点度数依次为 $d_1, d_2 ,\cdots,d_n$, 这样生成树的棵树为: $\dfrac{(n-2)!}{\prod (d_i-1)!}$
    \item 左边有 $n_1$ 个点, 右边有 $n_2$ 个点的完全二分图的生成树棵树为 $n_1^{n_2-1}\times n_2^{n_1-1}$
    \item $m$个连通块, 每个连通块有 $c_i$ 个点, 把他们全部连通的生成树方案数: $(\sum c_i)^{m-2} \prod c_i$
\end{itemize}

\subsubsection{无标号树计数}
\begin{enumerate}
    \item[(1)]有根树计数: $$f_n = \dfrac{ \sum_{i=1}^{n-1} f_{n-i} \sum_{d \mid i} f_d \cdot d}{n-1}$$
    
    记$g_i = \sum_{d \mid i} f_d \cdot d$ 即可做到$\Theta(n^2)$。
    \item[(2)]无根树计数:
    
    当 $n$ 是奇数时

    如果根不是重心,必然存在恰好一个子树,它的大小超过 $\left\lfloor\dfrac n2\right\rfloor$(设它的大小为 $k$)减去这种情况即可。
    
    因此答案为
    $$f_n-\sum_{k=\lfloor\frac n2\rfloor+1}^{n-1}f_k\cdot f_{n-k}$$
    
    当 $n$ 是偶数时
    
    有可能存在两个重心,且其中一个是根(即存在一棵子树大小恰为 $\dfrac n2$),额外减去$\dbinom{f_{\frac n2}}2$即可
\end{enumerate}

\subsubsection{有标号DAG计数}

$$
F_i=\sum_{j=1}^i \binom{i}{j}(-1)^{j+1}2^{j(i-j)}F_{i-j}
$$

想法是按照拓扑序分层,每次剥开所有入度为零的点。

\subsubsection{有标号连通简单图计数}

记$g(n)=2^{\binom n2}$为有标号简单图数量,$c(n)$为有标号简单连通图数量,那么枚举$1$所在连通块大小,有

$$
g(n) = \sum_{i=1}^n \dbinom{n-1}{i-1}c(i)g(n-i)
$$

易递推求$c(n)$。多项式做法考虑exp组合意义即可。

\subsubsection{生成树计数}
Kirchhoff Matrix $T = Deg - A$, $Deg$ 是度数对角阵, $A$ 是邻接矩阵. 

无向图度数矩阵是每个点度数; 有向图度数矩阵是每个点入度.

邻接矩阵 $A[u][v]$ 表示 $u \to v$ 边个数, 重边按照边数计算, 自环不计入度数.

无向图生成树计数: $c = |K$的任意$1$个 $n-1$ 阶主子式$|$

有向图外向树计数: $c = |$去掉根所在的那阶得到的主子式$|$

若求边权和则邻接矩阵可以设为$(1+wx)$,相当于一次项的系数。

\subsubsection{BEST定理}
设 $G$ 是有向欧拉图,$k$ 为任意顶点,那么 $G$ 的不同欧拉回路总数 $\mathrm{ec}(G)$ 是

$$
\mathrm{ec}(G) = t^\mathrm{root}(k)\prod_{v\in V}(\deg (v) - 1)!.
$$

$t^\mathrm{root}(k)$为以$k$为根的外向树个数。\clearpage\section{树论}
\subsection{快速LCA}
\noindent 查询$[dfn_u + 1 , dfn_v]$深度最小节点的父亲\\
可以简化为在 ST 表的最底层记录父亲,比较时取时间戳较小的结点。\\
取决于st表实现可以做到$O(n)$ or $O(n\log n)$预处理 $O(1)$查询
\begin{lstlisting}
int getmin(int x, int y) { return dfn[x] < dfn[y] ? x : y; }
void dfs(int u, int f) {
	dfn[u] = ++tim;
	a[dfn[u]] = f;  // TODO: build ST for a[i]
	for (int v : G[u])
		if (v != f) dfs(v, u);
}
int lca(int u, int v) {
	if (u == v) return u;
	if ((u = dfn[u]) > (v = dfn[v])) swap(u, v);
	return RMQ(dfn[u] + 1, dfn[v]);
}\end{lstlisting}
\subsection{虚树}
\begin{lstlisting}
vector<int> Gn[maxn];
int st[maxn], top;
void build(vector<int> v) {
	sort(v.beign(), v.end(),
	     [&](const int& a, const int& b) { return dfn[a] < dfn[b]; });
	top = 0;
	if (v[0] != 1) st[++top] = 1;  // Assume 1 is the root
	for (int u : v) {
		if (!top) {
			st[++top] = u;
			continue;
		}
		int anc = lca(st[top], u);
		if (anc == st[top]) {
			st[++top] = u;
			continue;
		}
		while (top > 1 && dfn[lca] <= dfn[st[top - 1]]) {
			Gn[st[top - 1]].pb(st[top]), top--;
		}
		if (anc != st[top]) Gn[anc].pb(st[top]), st[top] = anc;
		st[++top] = u;
	}
	while (top) Gn[st[top - 1]].pb(st[top]), top--;
}
// use DFS to clear Gn\end{lstlisting}
\subsection{长链剖分}
\subsubsection{优化dp}
\noindent 优化以深度为下标的树形DP\\
例如$dp(u,i)$表示$u$子树到达$u$距离为$i$的顶点信息,则考虑对于树进行长链剖分,$dfn_u$表示$u$在长链剖分的$dfn$序。\\
则可以将$dp(u,i)$记为$dp(dfn_u+i)$,就可以做到长链直接继承。
\subsubsection{k级祖先}
待补充

\subsection{静态点分治}
\begin{lstlisting}
void get_root(int u, int f) {
	sz[u] = 1, wt[u] = 0;
	for (int v : G[u]) {
		if (v == f || vis[v]) continue;
		get_root(v, u), sz[u] += sz[v], ckmax(wt[u], sz[v]);
	}
	ckmax(wt[u], Tsize - sz[u]);
	if (wt[Rt] > wt[u]) Rt = u;
}
void solve(int u) {
	vis[u] = 1;
	for (int v : G[u]) {
		if (vis[v]) continue;
		Rt = 0, Tsize = sz[v], get_root(v, 0);
		solve(Rt);
	}
}
wt[Rt = 0] = INF, Tsize = n;
get_root(1, 0);
solve(Rt);\end{lstlisting}
\subsection{点分树}
待验证,以下为邻域点权和模版(震波)
\begin{lstlisting}
void build(int u) {
	vis[u] = 1;
	t2[u].add(0, a[u]);
	for (int v : G[u]) {
		if (vis[v]) continue;
		Rt = 0, mxdep = 0, Tsize = sz[v];
		get_root(v, 0, 1);
		fa[Rt] = u;
		t1[Rt].init(mxdep + 5);
		t2[Rt].init(mxdep + 5);
		get_dis(v, u, 1);
		build(Rt);
	}
}
void modify(int u, int val) {
	for (int i = u; i; i = fa[i]) {
		t2[i].add(dis(u, i), val - a[u]);
		if (fa[i]) t1[i].add(dis(u, fa[i]), val - a[u]);
	}
	a[u] = val;
}
int query(int u, int k) {
	int rt = 0;
	for (int i = u; i; i = fa[i]) {
		rt += t2[i].query(k - dis(u, i));
		if (fa[i]) rt -= t1[i].query(k - dis(u, fa[i]));
	}
	return rt;
}\end{lstlisting}
\subsection{动态dp}
\begin{lstlisting}
void dfs1(int u) {
	siz[u] = 1;
	dep[u] = dep[fa[u]] + 1;
	for (int v : G[u]) {
		dfs1(v);
		siz[u] += siz[v];
		if (siz[v] > siz[son[u]]) son[u] = v;
	}
}
int endc[maxn];
Vector dp[maxn];  // F[u] 为 u 的 dp 值
Matrix trans[maxn];
// 考虑u点所有轻儿子以及u点点权的贡献转移矩阵,则某点u的dp值为 trans[u]*dp[son[u]]
void dfs2(int u, int t) {
	dfn[u] = ++tim, id[tim] = u;
	top[u] = t, endc[t] = max(endc[t], tim);
	// TODO: 初始化 F[u] 和 trans[u]
	if (son[u]) dfs2(son[u], t);
	for (int v : G[u]) {
		if (v == son[u]) continue;
		dfs2(v, v);
		// TODO: 用 dp[v] 更新 trans[u]
	}
	dp[u] = trans[u] * dp[son[u]];
}

struct Segtree {
	Matrix t[maxn << 2];
	void build(int u, int l, int r);  // t[u] = trans[id[x]];
	void pushup(int u);
	void update(int u, int l, int r, int x);  // t[u] = trans[id[x]]
	Matrix query(int u, int l, int r, int L, int R);
} T;

void update(int u) {
	// TODO: 更新 trans[u] 和 dp[u]
	Matrix aft;
	while (u != 0) {
		T.update(1, 1, n, dfn[u]);
		aft = T.query(1, 1, n, dfn[top[u]], endc[top[u]]);
		int v = top[u];
		u = fa[v];
		if (u) {}  // TODO: 用 aft 更新 trans[u] 和 dp[u]
	}
}
Vector query() { return T.query(1, 1, n, id[1], endc[1]) * dp[id[endc[1]]]; }\end{lstlisting}
\subsection{树上背包}
\begin{lstlisting}
// 背包大小上界为 m,复杂度为 O(nm)
void solve(int u) {
	sz[u] = 1;
	for (int v : G[u]) {
		solve(v);
		for (int i = 0; i <= m; i++) tmp[i] = 0;
		for (int i = 0; i <= min(m, sz[u]); i++)
			for (int j = 0; j <= min(m - i, sz[v]); j++)
				update(tmp[i + j], dp[u][i], dp[v][j]);
		sz[u] += sz[v];  // DON'T MOVE THIS!!!
		for (int i = 0; i <= m; i++) dp[u][i] = tmp[i];
	}
}\end{lstlisting}
\subsection{树哈希}
\begin{lstlisting}
mt19937_64 rnd(time(nullptr));
const ull mask = rnd();
const ull base = rnd();
ull xorshift(ull x) {
	x ^= mask;
	x ^= x << 13;
	x ^= x >> 7;
	x ^= x << 17;
	x ^= mask;
	return x;
}
ull hsh[maxn], sz[maxn];
void dfs(int u, int f) {
	hsh[u] = base, sz[u] = 1;
	for (int v : G[u]) {
		if (v == f) continue;
		dfs(v, u);
		hsh[u] += xorshift(hsh[v]);
		sz[u] += sz[v];
	}
	hsh[u] += xorshift(sz[u]);
}\end{lstlisting}
\clearpage\section{数论}
\subsection{数论分块}
\noindent 每一次 $[l,r]$ 都是 $n/l = n/r , m/l = m/r$ 的极大区间。\\
多个 $n, m$ 只要对多个 $n/(n/l)$ 取 $\min$ 即可,复杂度为$O(|cnt|\sqrt V)$
\begin{lstlisting}
for (ll l = 1, r = 1; l <= min(n, m); l = r + 1) {
	r = min(n / (n / l), m / (m / l));
	// Do something here
}\end{lstlisting}
\subsection{积性函数线性筛}
欧拉函数和莫比乌斯函数可以更简单的线性筛,见注释
\begin{lstlisting}
bool vis[maxn];
int prime[maxn], totp, mnpe[maxn], f[maxn];
void init() {
	vis[1] = 1;
	mnpe[1] = 1;  // mu[1] = ph[1] = 1
	for (int i = 2; i <= N; i++) {
		if (!vis[i])
			prime[++totp] = i, mnpe[i] = i;  // mu[i] = -1, phi[i] = i - 1;
		for (int j = 1; j <= totp && i * prime[j] <= N; j++) {
			vis[i * prime[j]] = 1;
			if (i % prime[j] == 0) {
				mnpe[i * prime[j]] = mnpe[i] * prime[j];
				// mu[i * prime[j]] = 0;
				// phi[i * prime[j]] = phi[i] * prime[j];
				break;
			}
			mnpe[i * prime[j]] = prime[j];
			// mu[i * prime[j]] = -mu[i];
			// phi[i * prime[j]] = phi[i] * (prime[j] - 1);
		}
	}
	for (int i = 1; i <= totp; i++)
		for (int e = 1, p = prime[i]; p <= N; e++, p *= prime[i]) {
			// TODO: 在这里计算素数幂处的值 f[p]
		}
	for (int i = 1; i <= N; i++)
		if (i != mnpe[i]) f[i] = f[mnpe[i]] * f[i / mnpe[i]];
}\end{lstlisting}
\subsection{筛子}
\subsubsection{杜教筛}
若想要求出$f$在$n$处的前缀和$s(n) = \sum\limits_{i=1}^n f(i)$,构造积性函数$g$,设$h =  f \ast g$,则
$$
\begin{aligned}
& \sum\limits_{i = 1} ^ n h(i) \\
= \ & \sum\limits_{ij \leq n} f(i) g(j) \\
= \ & \sum\limits_{d = 1} ^ n g(d) \sum\limits_{i = 1} ^ {\frac n d} f(i) \\ 
= \ & \sum\limits_{d = 1} ^ n g(d) s\left(\left\lfloor\frac n d\right\rfloor\right)
\end{aligned}
$$

若$g,h$的前缀和可以快速求出,则

$$
s(n) = \sum\limits_{i = 1} ^ n h(i) - \sum\limits_{d = 2} ^ n g(d)s\left(\left\lfloor\frac n d\right\rfloor\right)
$$

预处理$f$的前缀和到$n^{2/3}$处即可做到单次查询$O(n^{2/3})$。

\begin{lstlisting}
ll sum_f(ll x) {
	if (x <= N) return sf[x];
	if (Sf[n / x]) return Sf[n / x];
	ll ans = Sum_h(x);
	for (ll l = 2, r; l <= x; l = r + 1) {
		r = x / (x / l);
		sub(ans, sum_g(l, r) * sum_f(x / l));
	}
	return Sf[n / x] = ans;
}
\end{lstlisting}
\subsubsection{min-25筛(质数个数)}
\begin{lstlisting}
const int N = 1e6;
double inv[maxn];
void sieve() {
	// 见线性筛部分
	for (int i = 1; i <= N; ++i) inv[i] = 1.0 / i;
}
ll val[maxn], pos[maxn], id1[maxn], id2[maxn], cnt;
ll solve(ll n) {
	ll T = sqrt(n) + 3;
	for (ll i = 1, pi; i <= n; i = pi + 1) {
		pos[++cnt] = n / i;
		pi = n / pos[cnt];
		val[cnt] = pos[cnt] - 1;
		(n / i <= T ? id1[n / i] : id2[pi]) = cnt;
	}
	auto getid = [&](ll x) -> int { return x <= T ? id1[x] : id2[n / x]; };
	for (int i = 1; 1ll * prime[i] * prime[i] <= n; i++) {
		int p = id1[prime[i - 1]];
		for (int j = 1; pos[j] >= 1ll * prime[i] * prime[i]; j++) {
			int q = getid(1.0 * pos[j] * inv[prime[i]] + 1e-7);
			val[j] -= val[q] - val[p];
		}
	}
	return val[1];
}\end{lstlisting}
\subsubsection{min-25筛}
未验证,待补充
\begin{lstlisting}
void init_g() {
	for (ll l = 1, r; l <= n; l = r + 1) {
		r = n / (n / l);
		w[++tot] = n / l;
		g ^ k[tot] = sum(1..w [tot] ^ k);
		if (n / l <= N) ind1[n / l] = tot;
		else ind2[l] = tot;
	}
}
void calc_g() {
	for (int i = 1; i <= totp; i++) {
		for (int j = 1; j <= tot && prime[i] * prime[i] <= w[j]; j++) {
			ll k = w[j] / prime[i] <= N ? ind1[w[j] / prime[i]]
			                            : ind2[n / (w[j] / prime[i])];
			sub(g ^ e[j], prime[i] ^ e * (g ^ e[k] - sp ^ e[i - 1]));
		}
	}
}
ll S(ll x, int y) {
	cnt++;
	if (prime[y] >= x) return 0;
	ll k = x <= N ? ind1[x] : ind2[n / x];
	ll ans = f(g ^ e(x)) - f(sp ^ e[y]);
	for (int i = y + 1; i <= totp && prime[i] * prime[i] <= x; i++) {
		ll p = prime[i];
		for (int e = 1; p <= x; e++, p *= prime[i]) {
			add(ans, f(p) * (S(x / p, i) + (e != 1)));
		}
	}
	return ans % MOD;
}\end{lstlisting}
\subsubsection{PowerfulNumber筛}
未验证,待补充。
\par 求$\sum_{i=1}^n f(i)$,找到$g(i)$满足$\forall p\in
\mathbf{P},f(p)=g(p)$,g可杜教筛。 \par 不妨假设 $g\ast h =
f$,那么$h$仅在PN处有值,于是
$$\begin{aligned} \sum_{i=1}^{n} f(i)&=\sum_{ij} g(i)h(j) \\ &= \sum_{i\le n}
h(i) \sum_{j=1}^{n/i} g(j)\end{aligned}$$ \par 计算$h$依靠递推:
$$\begin{aligned}
f(p^k)&=\sum_{i+j=k}g(p^i)h(p^j)\\ h(p^k)&=f(p^k)-\sum_{i=0}^{k-1}h(p^i)g(p^{k-i})\end{aligned}$$
\begin{lstlisting}

void power_num(int k, ll m, int h) {
	if (k > totp || m * prime[k] > n) {
		if (n / m <= N) add(ans, 1ll * h * sxphi[n / m] % MOD);
		else add(ans, 1ll * h * Sxphi[n / (n / m)] % MOD);
		return;
	}
	power_num(k + 1, m, h);
	ll p = 1ll * prime[k] * prime[k];
	for (int e = 2; m * p <= n; p *= prime[k], e++) {
		power_num(k + 1, m * p,
		          p % MOD * (prime[k] - 1) % MOD * (e - 1) % MOD * h % MOD);
	}
}\end{lstlisting}
\subsubsection{洲阁筛}
我不会,长大后再学习。\subsection{扩展欧几里得}
\begin{lstlisting}
ll exgcd(ll a, ll b, ll& x, ll& y) {
	if (!b) return x = 1, y = 0, a;
	else {
		ll rt = exgcd(b, a % b, y, x);
		y -= (a / b) * x;
		return rt;
	}
}\end{lstlisting}
\subsection{欧拉定理}
当$(a,m)=1$时,
$$a^{\varphi(m)}\equiv 1(\bmod\ m)$$。

当$(a,m)\neq 1$时,

$$a^b\equiv a^{\min\{b, b \bmod\ \varphi(m)+\varphi(m)\}}(\bmod\ m)$$。\subsection{中国剩余定理}
解方程:
\[
    \begin{cases}
        x\equiv a_1(\bmod\ m_1)\\
        x\equiv a_2(\bmod\ m_2)\\
        \vdots\\
        x\equiv a_n(\bmod\ m_n)
    \end{cases}
\]
若 $m_i$ 两两互质,则可以使用以下公式得到:
\[
\begin{aligned}
    x \equiv \sum_{i=1}^n M_i \times N_i \times a_i (\bmod\ M)\\
    \text{where:}
    \begin{cases}
        M = \prod_{i=1}^n m_i \\
        M_i = \frac{M}{m_i}\\
        N_i \times M_i \equiv 1 (\bmod\ m_i)
    \end{cases}
\end{aligned}
\]
否则参考以下exCRT。
\begin{lstlisting}
ll exCRT(vector<int> a, vector<int> m) {
	assert(a.size() == m.size());
	ll ans = a[0], M = b[0];
	for (int i = 1; i < a.size(); i++) {
		ll x = 0, y = 0;
		ll A = M, B = m[i], C = (a[i] - ans % B + B) % B;
		ll gcd = exgcd(A, B, x, y), bg = B / gcd;
		x = x * (C / gcd) % B;
		ans += x * M;
		M *= bg;
		ans = (ans % M + M) % M;
	}
	return (ans % M + M) % M;
}
\end{lstlisting}
\subsection{BSGS}
\begin{lstlisting}
gp_hash_table<ll, ll> s;
ll exgcd(ll a, ll b) {
	if (a == 1) return 1;
	return (1 - b * exgcd(b % a, a)) / a;  // not ll
}
ll exBSGS(ll a, ll b, ll p) {
	s.clear();
	a %= p, b %= p;
	ll j = 1 % p, cnt = 0;
	for (int i = 0; i <= __lg(p); i++, j = j * a % p)
		if (j == b) return i;
	ll x, y = 1;
	while (true) {
		x = gcd(a, p);
		if (x == 1) break;
		if (b % x) return -1;  // no sol
		cnt++;
		p /= x, b /= x;
		y = y * (a / x) % p;
	}
	a %= p;
	b = (ll)b * (p + exgcd(y, p)) % p;
	x = ceil(sqrt(p)), j = 1;
	for (int i = 0; i < x; i++, j = j * a % p) {
		if (j == b) return i + cnt;
		s[j * b % p] = i + 1;
	}
	int k = j;
	for (int i = 1; i <= x; i++, j = (ll)j * k % p)
		if (s[j]) return (ll)i * x + cnt - s[j] + 1;
	return -1;
}\end{lstlisting}
\subsection{Millar-Robin}
Pollard-Rho 的 $2\sim 30$ 行即为 Millar-Robin\subsection{Pollard-Rho}
\begin{lstlisting}
namespace factor {
	using f64 = long double;
	ll p;
	f64 invp;
	inline void setmod(ll x) { p = x, invp = (f64)1 / x; }
	inline ll mul(ll a, ll b) {
		ll z = a * invp * b + 0.5;
		ll res = a * b - z * p;
		return res + (res >> 63 & p);
	}
	inline ll pow(ll a, ll x, ll res = 1) {
		for (; x; x >>= 1, a = mul(a, a))
			if (x & 1) res = mul(res, a);
		return res;
	}
	inline bool checkprime(ll p) {
		if (p == 1) return 0;
		setmod(p);
		ll d = __builtin_ctzll(p - 1), s = (p - 1) >> d;
		for (ll a : {2, 3, 5, 7, 11, 13, 82, 373}) {
			if (a % p == 0) continue;
			ll x = pow(a, s), y;
			for (int i = 0; i < d; ++i, x = y) {
				y = mul(x, x);
				if (y == 1 && x != 1 && x != p - 1) return 0;
			}
			if (x != 1) return 0;
		}
		return 1;
	}
	inline ll rho(ll n) {
		if (!(n & 1)) return 2;
		static std::mt19937_64 gen((size_t) "hehezhou");
		ll c = gen() % (n - 1) + 1, y = gen() % (n - 1) + 1;
		auto f = [&](ll o) {
			o = mul(o, o) + c;
			return o >= n ? o - n : o;
		};
		setmod(p);
		for (int l = 1;; l <<= 1) {
			ll x = y, g = 1;
			for (int i = 0; i < l; ++i) y = f(y);
			const int d = 512;
			for (int i = 0; i < l; i += d) {
				ll sy = y;
				for (int j = 0; j < min(d, l - i); ++j) {
					y = f(y), g = mul(g, (y - x + n));
				}
				g = gcd(n, g);
				if (g == 1) continue;
				if (g == n)
					for (g = 1, y = sy; g == 1;)
						y = f(y), g = gcd(n, y - x + n);
				return g;
			}
		}
	}
	inline std::vector<ll> factor(ll x) {
		std::queue<ll> q;
		q.push(x);
		std::vector<ll> res;
		for (; q.size();) {
			ll x = q.front();
			q.pop();
			if (x == 1) continue;
			if (checkprime(x)) {
				res.push_back(x);
				continue;
			}
			ll y = rho(x);
			q.push(y), q.push(x / y);
		}
		sort(res.begin(), res.end());
		return res;
	}
}  // namespace factor\end{lstlisting}
\subsection{原根}
你说的对,但是感觉不如原根。
\par 原根,是一个数学符号。
设 $m$ 是正整数,$a$是整数,若 $a$ 模 $m$ 的阶等于 $\varphi(m)$
(定义 $a$ 模 $m$ 的阶 $\delta_m(a)$ 为最小的$x$ 满足 $a^x\equiv 1 \pmod m$),则称 $a$ 为模 $m$ 的一个原根。
\par 假设一个数 $g$ 是 $p\in\textbf P$ 的原根,那么$\forall 0<i<p,g^i \bmod p$ 的结果两两不同,
归根到底就是 $g^a \equiv 1 \pmod p$ 当且仅当指数 $a$ 为 $p-1$ 的倍数时成立。
\par 你的数学很差,我现在每天用原根都能做 $10^5$ 次数据规模 $10^6$ 的NTT,
每个月差不多 $3\times10^6$ 次卷积,即 $2\times10^6$次常系数齐次线性递推,
也就是现实生活中$6.4\times10^{19}$ 次乘法运算,换算过来最少也要算 $2\times10^4$年。
虽然我只有 $14$岁,但是已经超越了中国绝大多数人(包括你)的水平,这便是原根给我的骄傲的资本。



性质:

\begin{itemize}
    \item 最小原根大小数量级在$O(m^{1/4})$左右,求最小原根直接枚举$g$并检验 对于 $\varphi(m)$ 的每个素因数 $p$,都有 $g^{\frac{\varphi(m)}{p}}\not\equiv 1\pmod m$ 即可。
    \item 
    $$
    \delta_m(a^k)=\dfrac{\delta_m(a)}{\left(\delta_m(a),k\right)}
    $$
    通常这里的$a$是$g$。
    \item 将模$m$剩余系看成一个$\times g$的循环群。
\end{itemize}\subsection{二次剩余魔法}
待补充
\subsection{类欧几里得和万能欧几里得}
未验证,待补充
板子为直线下点数:
$$ans = \sum_{i=1}^{n-1}\lfloor\frac{ai+b}{m}\rfloor $$
\begin{lstlisting}
ull floor_sum(ull n, ull m, ull a, ull b) {
	ull ans = 0;
	while (true) {
		if (a >= m) ans += n * (n - 1) / 2 * (a / m), a %= m;
		if (b >= m) ans += n * (b / m), b %= m;
		ull ymax = a * n + b;  // use u128 if it's big
		if (ymax < m) break;
		n = ymax / m;
		b = ymax % m;
		std::swap(m, a);
	}
	return ans;
}\end{lstlisting}
\subsection{Lucas和扩展Lucas}
$$ \dbinom{n}{m}\equiv\dbinom{n_1}{m_1}\dbinom{n_2}{m_2}\dbinom{n_3}{m_3}\cdots \pmod p$$

推论:$\dbinom{n}{m}\not\equiv 0 \pmod p​$ 当且仅当$m$和$n-m$在$p$进制相加没有进位。例如$p=2$时只有$n\land m=m$时$\dbinom{n}{m}\equiv1(mod\ 2)$。

扩展Lucas待补充。\clearpage\section{数学}
\subsection{矩阵}
\subsection{多项式合集}
务必记得\texttt{Init(n)}在进行多项式乘法前!!!
\begin{lstlisting}
const int MOD = 998244353;
const int N = 20;
const int G = 3;
const int MAXN = 3e5 + 5;
typedef vector<int> poly;
void print(const poly& a) {
	for (int i = 0; i < (int)a.size(); i++)
		cout << (i == 0 ? " " : "+ ") << a[i] << "*x^" << i << ' ';
	cout << endl;
}
namespace Poly {
	const int mod = 998244353;
	int rev[MAXN], w[MAXN], wn[N];
	void addmod(int& x, int y) {
		x += y;
		if (x >= mod) x -= mod;
	}
	void submod(int& x, int y) {
		x -= y;
		if (x < 0) x += mod;
	}
	int add(int x, int y) {
		addmod(x, y);
		return x;
	}
	int sub(int x, int y) {
		submod(x, y);
		return x;
	}
	int power(int x, int y) {
		int res = 1;
		while (y) {
			if (y & 1) res = (ll)res * x % mod;
			x = (ll)x * x % mod;
			y >>= 1;
		}
		return res;
	}
	int Inv(int x) { return power(x, mod - 2); }
	void InitNTT(int n) {
		wn[n] = power(G, (mod - 1) / (1 << n));
		for (int i = n - 1; i >= 0; i--)
			wn[i] = (ll)wn[i + 1] * wn[i + 1] % mod;
	}
	int Init(int n) {
		int len = 1;
		while (len < n) len <<= 1;
		for (int i = 0; i < len; i++)
			rev[i] = (rev[i >> 1] >> 1) | ((i & 1) * (len >> 1));
		for (int i = 1, t = 1; i < len; i <<= 1, t += 1) {
			w[i] = 1;
			for (int j = 1; j < i; j++)
				w[i + j] = (ll)w[i + j - 1] * wn[t] % mod;
		}
		return len;
	}
	void NTT(poly& a, int flag) {
		int n = a.size();
		for (int i = 0; i < n; i++)
			if (i < rev[i]) swap(a[i], a[rev[i]]);
		for (int i = 2; i <= n; i <<= 1) {
			int mid = (i >> 1);
			for (int j = 0; j < n; j += i) {
				for (int k = j; k < j + mid; k++) {
					int x = a[k], y = (ll)a[k + mid] * w[k - j + mid] % mod;
					a[k] = add(x, y);
					a[k + mid] = sub(x, y);
				}
			}
		}
		if (flag == -1) {
			reverse(a.begin() + 1, a.begin() + n);
			int invn = Inv(n);
			for (int i = 0; i < n; i++) a[i] = (ll)a[i] * invn % mod;
		}
	}
	poly PolyAdd(const poly& A, const poly& B) {
		poly res = A;
		for (int i = 0; i < (int)A.size(); i++) addmod(res[i], B[i]);
		return res;
	}
	poly PolyMul(const poly& A, const poly& B, int need = 0) {
		int n = A.size(), m = B.size();
		if (n < 5 || m < 5) {
			poly a;
			a.resize(n + m - 1);
			for (int i = 0; i < n; i++)
				for (int j = 0; j < m; j++)
					addmod(a[i + j], (ll)A[i] * B[j] % mod);
			if (need) a.resize(need);
			return a;
		}
		int len = Init(n + m);
		poly a = A, b = B;
		a.resize(len), b.resize(len);
		NTT(a, 1);
		NTT(b, 1);
		for (int i = 0; i < len; i++) a[i] = (ll)a[i] * b[i] % mod;
		NTT(a, -1);
		a.resize(need ? need : n + m - 1);
		return a;
	}
	poly PolyInv(const poly& A) {
		int n = A.size();
		if (n == 1) { return {Inv(A[0])}; }
		poly a = A, b = PolyInv(poly(A.begin(), A.begin() + ((n + 1) >> 1)));
		int len = Init(n << 1);
		a.resize(len), b.resize(len);
		NTT(a, 1);
		NTT(b, 1);
		for (int i = 0; i < len; i++)
			b[i] = (ll)sub(2, (ll)a[i] * b[i] % mod) * b[i] % mod;
		NTT(b, -1);
		b.resize(n);
		return b;
	}
	poly PolyDeriv(const poly& A) {
		int n = A.size();
		poly a = A;
		for (int i = 1; i < n; i++) a[i - 1] = (ll)i * A[i] % mod;
		a[n - 1] = 0;
		return a;
	}
	poly PolyInter(const poly& A) {
		int n = A.size();
		poly a = A;
		for (int i = 1; i < n; i++)
			a[i] = (ll)A[i - 1] * power(i, mod - 2) % mod;
		a[0] = 0;
		return a;
	}
	pair<poly, poly> PolyMod(const poly& A, const poly& B) {
		int n = A.size(), m = B.size();
		if (n < m) return make_pair(poly(1), A);
		poly a = A, b = B;
		reverse(a.begin(), a.end());
		reverse(b.begin(), b.end());
		b.resize(n - m + 1);
		b = PolyInv(b);
		a.resize(n - m + 1);
		a = PolyMul(a, b, n - m + 1);
		reverse(a.begin(), a.end());
		b = PolyMul(a, B, m - 1);
		for (int i = 0; i < m - 1; i++) b[i] = sub(A[i], b[i]);
		return make_pair(a, b);
	}
	poly PolyLn(const poly& A) {
		int n = A.size();
		poly a = A;
		for (int i = 1; i < n; i++) a[i - 1] = (ll)i * A[i] % mod;
		a[n - 1] = 0;
		a = PolyMul(a, PolyInv(A), n);
		for (int i = n - 1; i >= 1; i--)
			a[i] = (ll)a[i - 1] * power(i, mod - 2) % mod;
		a[0] = 0;
		return a;
	}
	poly PolyExp(const poly& A) {
		int n = A.size();
		if (n == 1) return {1};
		poly b = PolyExp(poly(A.begin(), A.begin() + ((n + 1) >> 1)));
		b.resize(n);
		poly c = PolyLn(b);
		for (int i = 0; i < n; i++) c[i] = sub(A[i], c[i]);
		addmod(c[0], 1);
		poly d = PolyMul(b, c, n);
		return d;
	}
	poly PolyPow(const poly& A, int k) {
		int n = A.size();
		poly a;
		a.resize(n);
		if (!k) {
			a[0] = 1;
			return a;
		}
		int p = 0;
		while (p < n && !A[p]) p += 1;
		if ((ll)p * k >= n) return a;
		int m = n - p * k;
		a.resize(m);
		int coef = power(A[p], k), icoef = power(A[p], mod - 2);
		for (int i = 0; i < m; i++) a[i] = (ll)A[i + p] * icoef % mod;
		a = PolyLn(a);
		for (int i = 0; i < m; i++) a[i] = (ll)a[i] * k % mod;
		a = PolyExp(a);
		poly b;
		b.resize(n);
		for (int i = 0; i < m; i++) b[i + p * k] = (ll)a[i] * coef % mod;
		return b;
	}
	poly tmp[MAXN];
#define lson k << 1
#define rson k << 1 | 1
	void pre_eval(const poly& A, int k, int l, int r) {
		if (l == r) {
			tmp[k].resize(2);
			tmp[k][0] = sub(0, A[l]);
			tmp[k][1] = 1;
			return;
		}
		int mid = (l + r) >> 1;
		pre_eval(A, lson, l, mid);
		pre_eval(A, rson, mid + 1, r);
		tmp[k] = PolyMul(tmp[lson], tmp[rson]);
	}
	void solve_eval(const poly& A, const poly& B, poly& C, int k, int l,
	                int r) {
		if (r - l <= 30) {
			for (int i = l; i <= r; i++)
				for (int j = A.size() - 1; j >= 0; j--)
					C[i] = (A[j] + (ll)C[i] * B[i] % mod) % mod;
			return;
		}
		int mid = (l + r) >> 1;
		solve_eval(PolyMod(A, tmp[lson]).second, B, C, lson, l, mid);
		solve_eval(PolyMod(A, tmp[rson]).second, B, C, rson, mid + 1, r);
	}
	poly PolyEval(const poly& A, const poly& B) {
		int m = B.size();
		pre_eval(B, 1, 0, m - 1);
		poly c;
		c.resize(m);
		solve_eval(PolyMod(A, tmp[1]).second, B, c, 1, 0, m - 1);
		return c;
	}
	poly solve_itpl(const poly& A, int k, int l, int r) {
		if (l == r) return {A[l]};
		int mid = (l + r) >> 1;
		return PolyAdd(PolyMul(solve_itpl(A, lson, l, mid), tmp[rson]),
		               PolyMul(solve_itpl(A, rs on, mid + 1, r), tmp[lson]));
	}
	poly PolyItpl(const poly& A, const poly& B) {
		int n = A.size();
		pre_eval(A, 1, 0, n - 1);
		poly a;
		a.resize(n);
		solve_eval(PolyDeriv(tmp[1]), A, a, 1, 0, n - 1);
		for (int i = 0; i < n; i++) a[i] = (ll)B[i] * Inv(a[i]) % MOD;
		return solve_itpl(a, 1, 0, n - 1);
	}
#undef lson
#undef rson
	struct Initializer {
		Initializer() { InitNTT(N - 1); }
	} initializer;
}  // namespace Poly\end{lstlisting}
\subsection{BM}
$$\forall i, \sum_{j=0}^m a_{i-j}v_j = 0$$
\begin{lstlisting}
vector<int> berlekamp_massey(const vector<int>& a) {
	vector<int> v, last;  // v is the answer, 0-based
	int k = -1, delta = 0;
	for (int i = 0; i < (int)a.size(); i++) {
		int tmp = 0;
		for (int j = 0; j < (int)v.size(); j++)
			tmp = (tmp + (long long)a[i - j - 1] * v[j]) % p;
		if (a[i] == tmp) continue;
		if (k < 0) {
			k = i;
			delta = (a[i] - tmp + p) % p;
			v = vector<int>(i + 1);
			continue;
		}
		vector<int> u = v;
		int val = (long long)(a[i] - tmp + p) * qpow(delta, p - 2) % p;
		if (v.size() < last.size() + i - k) v.resize(last.size() + i - k);
		(v[i - k - 1] += val) %= p;
		for (int j = 0; j < (int)last.size(); j++) {
			v[i - k + j] = (v[i - k + j] - (long long)val * last[j]) % p;
			if (v[i - k + j] < 0) v[i - k + j] += p;
		}
		if ((int)u.size() - i < (int)last.size() - k) {
			last = u;
			k = i;
			delta = a[i] - tmp;
			if (delta < 0) delta += p;
		}
	}
	for (auto& x : v) x = (p - x) % p;
	v.insert(v.begin(), 1);  //一般是需要最小递推式的, 处理一下
	return v;
}
\end{lstlisting}
\subsection{线性规划单纯形法}
\begin{lstlisting}
using db = long double;
const db eps = 1e-16;
int sgn(db x) { return x < -eps ? -1 : x > eps; }
namespace LP {
	const int N = 21, M = 21;
	int n, m;  // n : 变量个数,m : 约束个数
	db a[M + N][N], x[N + M];
	// 约束:对于 1 <= i <= m : a[i][0] + \sum_j x[j] * a[i][j] >= 0
	// x[j] >= 0
	// 最大化 \sum_j x[j] * a[0][j]
	int id[N + M];
	void pivot(int p, int o) {
		std::swap(id[p], id[o + n]);
		db w = -a[o][p];
		for (int i = 0; i <= n; ++i) a[o][i] /= w;
		a[o][p] = -1 / w;
		for (int i = 0; i <= m; ++i)
			if (sgn(a[i][p]) && i != o) {
				db w = a[i][p];
				a[i][p] = 0;
				for (int j = 0; j <= n; ++j) a[i][j] += w * a[o][j];
			}
	}
	db solve() {  // nan : 无解,inf : 无界,否则返回最大值
		for (int i = 1; i <= n + m; ++i) id[i] = i;
		for (;;) {
			int p = 0, min = 1;
			for (int i = 1; i <= m; ++i) {
				if (a[i][0] < a[min][0]) min = i;
			}
			if (a[min][0] >= -eps) break;
			for (int i = 1; i <= n; ++i)
				if (a[min][i] > eps && id[i] > id[p]) { p = i; }
			if (!p) return nan("");
			pivot(p, min);
		}
		for (;;) {
			int p = 1;
			for (int i = 1; i <= n; ++i)
				if (a[0][i] > a[0][p]) p = i;
			if (a[0][p] < eps) break;
			db min = INFINITY;
			int o = 0;
			for (int i = 1; i <= m; ++i)
				if (a[i][p] < -eps) {
					db w = -a[i][0] / a[i][p];
					int d = sgn(w - min);
					if (d < 0 || !d && id[i] > id[o]) o = i, min = w;
				}
			if (!o) return INFINITY;
			pivot(p, o);
		}
		for (int i = 1; i <= m; ++i) x[id[i + n]] = a[i][0];
		return a[0][0];
	}
}  // namespace LP\end{lstlisting}
\subsection{FWT}
$$C_i = \sum_{j\oplus k = i}A_j\times B_k$$。
$A$, $B$ FWT后对应位相乘在iFWT回去。
\begin{lstlisting}
// op = 1 / -1
inline void FMT_OR(int* a, int n, int op) {
	for (int i = 0; i < n; i++)
		for (int j = 0; j < (1 << n); j++)
			if ((1 << i) & j)
				add(a[j], op == 1 ? a[j ^ (1 << i)] : MOD - a[j ^ (1 << i)]);
}
inline void FMT_AND(int* a, int n, int op) {
	for (int i = 0; i < n; i++)
		for (int j = (1 << n) - 1; j >= 0; j--)
			if (!((1 << i) & j))
				add(a[j], op == 1 ? a[j ^ (1 << i)] : MOD - a[j ^ (1 << i)]);
}
const int inv2 = 499122177;
inline void FWT_XOR(int* a, int n, int op) {
	for (int i = 1; i < (1 << n); i <<= 1)
		for (int j = 0; j < (1 << n); j += (i << 1))
			for (int k = 0; k < i; k++) {
				int t = a[j + k];
				a[j + k] = mod1(t + a[i + j + k]);
				a[i + j + k] = mod1(t + MOD - a[i + j + k]);
				if (~op) {
					a[j + k] = 1ll * a[j + k] * inv2 % MOD;
					a[i + j + k] = 1ll * a[i + j + k] * inv2 % MOD;
				}
			}
}\end{lstlisting}
\subsection{常见数和公式}
待补充。

\begin{itemize}
    \item 一般幂转下降幂: $$x^n=\sum\limits_{i=0}^n\begin{Bmatrix}n\\i\end{Bmatrix}x^{\underline{i}},n\geq 0$$
\end{itemize}\clearpage\section{字符串}
Delete This
\begin{lstlisting}
\end{lstlisting}
\clearpage\section{数据结构}
Delete This
\begin{lstlisting}
\end{lstlisting}
\clearpage\section{计算几何}
\subsection{计算几何}
by yhx-12243,暂未验证
\begin{lstlisting}
const double eps = 1e-8;

#define lt(x, y) ((x) < (y)-eps)
#define gt(x, y) ((x) > (y) + eps)
#define le(x, y) ((x) <= (y) + eps)
#define ge(x, y) ((x) >= (y)-eps)
#define eq(x, y) (le(x, y) && ge(x, y))
#define dot(x, y, z) (((y) - (x)) * ((z) - (x)))
#define cross(x, y, z) (((y) - (x)) ^ ((z) - (x)))
struct vec2 {
	double x, y;
	vec2(double x0 = 0.0, double y0 = 0.0) : x(x0), y(y0) {}
	vec2* read() {
		scanf("%lf%lf", &x, &y);
		return this;
	}
	inline vec2 operator-() const { return vec2(-x, -y); }
	inline vec2 operator+(const vec2& B) const {
		return vec2(x + B.x, y + B.y);
	}
	inline vec2 operator-(const vec2& B) const {
		return vec2(x - B.x, y - B.y);
	}
	inline vec2 operator*(double k) const { return vec2(x * k, y * k); }
	inline vec2 operator/(double k) const { return *this * (1.0 / k); }
	inline double operator*(const vec2& B) const { return x * B.x + y * B.y; }
	inline double operator^(const vec2& B) const { return x * B.y - y * B.x; }
	inline double norm2() const { return x * x + y * y; }
	inline double norm() const { return sqrt(x * x + y * y); }
	inline bool operator<(const vec2& B) const {
		return lt(x, B.x) || le(x, B.x) && lt(y, B.y);
	}
	inline bool operator==(const vec2& B) const {
		return eq(x, B.x) && eq(y, B.y);
	}
	inline bool operator<<(const vec2& B) const {
		return lt(y, 0) ^ lt(B.y, 0)
		           ? lt(B.y, 0)
		           : gt(*this ^ B, 0)
		                 || ge(*this ^ B, 0) && ge(x, 0) && lt(B.x, 0);
	}
	inline vec2 trans(double a11, double a12, double a21, double a22) const {
		return vec2(x * a11 + y * a12, x * a21 + y * a22);
	}
};
/*
operator * : Dot product
operator ^ : Cross product
norm2() : |v|^2 = v.v
norm() : |v| = sqrt(v.v)
operator < : Two-key compare
operator << : Polar angle compare
trans : Transition with a 2x2 matrix
*/
struct line {
	double A, B, C;  // Ax + By + C = 0, > 0 if it represents halfplane.
	line(double A0 = 0.0, double B0 = 0.0, double C0 = 0.0)
	    : A(A0), B(B0), C(C0) {}
	line(const vec2& u, const vec2& v)
	    : A(u.y - v.y), B(v.x - u.x), C(u ^ v) {}  // left > 0
	inline vec2 normVec() const { return vec2(A, B); }
	inline double norm2() const { return A * A + B * B; }
	inline double operator()(const vec2& P) const {
		return A * P.x + B * P.y + C;
	}
};
inline vec2 intersection(const line u, const line v) {
	return vec2(u.B * v.C - u.C * v.B, u.C * v.A - u.A * v.C)
	       / (u.A * v.B - u.B * v.A);
}
inline bool parallel(const line u, const line v) {
	double p = u.normVec() ^ v.normVec();
	return eq(p, 0);
}
inline bool perpendicular(const line u, const line v) {
	double p = u.normVec() * v.normVec();
	return eq(p, 0);
}
inline bool sameDir(const line u, const line v) {
	return parallel(u, v) && gt(u.normVec() * v.normVec(), 0);
}
inline line bisector(const vec2 u, const vec2 v) {
	return line(v.x - u.x, v.y - u.y, 0.5 * (u.norm2() - v.norm2()));
}
inline double dis2(const vec2 P, const line l) {
	return l(P) * l(P) / l.norm2();
}
inline vec2 projection(const vec2 P, const line l) {
	return P - l.normVec() * (l(P) / l.norm2());
}
inline vec2 symmetry(const vec2 P, const line l) {
	return P - l.normVec() * (2 * l(P) / l.norm2());
}
// Relation of 3 points. (2 inside, 1 outside, 0 not collinear)
inline int collinear(const vec2 u, const vec2 v, const vec2 P) {
	double p = cross(P, u, v);
	return eq(p, 0) ? 1 + le(dot(P, u, v), 0) : 0;
}
// Perimeter of a polygon
double perimeter(int n, vec2* poly) {
	double ret = (poly[n - 1] - *poly).norm();
	for (int i = 1; i < n; ++i) ret += (poly[i - 1] - poly[i]).norm();
	return ret;
}
// Directed area of a polygon (positive if CCW)
double area(int n, vec2* poly) {
	double ret = poly[n - 1] ^ *poly;
	for (int i = 1; i < n; ++i) ret += poly[i - 1] ^ poly[i];
	return ret * 0.5;
}
// Point in polygon (2 on boundary, 1 inside, 0 outside)
int contain(int n, vec2* poly, const vec2 P) {
	int in = 0;
	vec2 *r = poly + (n - 1), *u, *v;
	for (int i = 0; i < n; r = poly, ++poly, ++i) {
		if (collinear(*r, *poly, P) == 2) return 2;
		gt(r->y, poly->y) ? (u = poly, v = r) : (u = r, v = poly);
		if (ge(P.y, v->y) || lt(P.y, u->y)) continue;
		in ^= gt(cross(P, *u, *v), 0);
	}
	return in;
}
// Convex Hall
int graham(int n, vec2* p, vec2* dest) {
	int i;
	vec2* ret = dest;
	std::iter_swap(p, std::min_element(p, p + n));
	std::sort(p + 1, p + n, [p](const vec2 A, const vec2 B) {
		double r = cross(*p, A, B);
		return gt(r, 0) || (ge(r, 0) && lt((A - *p).norm2(), (B - *p).norm2()));
	});
	for (i = 0; i < 2 && i < n; ++i) *ret++ = p[i];
	for (; i < n; *ret++ = p[i++])
		for (; ret != dest + 1 && ge(cross(ret[-2], p[i], ret[-1]), 0); --ret)
			;
	return *ret = *p, ret - dest;
}
double convDiameter(int n, vec2* poly) {
	int l = std::min_element(poly, poly + n) - poly,
	    r = std::max_element(poly, poly + n) - poly, i = l, j = r;
	double diam = (poly[l] - poly[r]).norm2();
	do {
		(ge(poly[(i + 1) % n] - poly[i] ^ poly[(j + 1) % n] - poly[j], 0)
		     ? ++j
		     : ++i) %= n;
		up(diam, (poly[i] - poly[j]).norm2());
	} while (i != l || j != r);
	return diam;
}
//
inline vec2 circumCenter(const vec2 A, const vec2 B, const vec2 C) {
	vec2 a = B - A, b = C - A, AO;
	double det = 0.5 / (a ^ b), na = a.norm2(), nb = b.norm2();
	AO = vec2((na * b.y - nb * a.y) * det, (nb * a.x - na * b.x) * det);
	return A + AO;
}
double minCircleCover(int n, vec2* p, vec2* ret = NULL) {
	int i, j, k;
	double r2 = 0.0;
	std::random_shuffle(p + 1, p + (n + 1));
	vec2 C = p[1];
	for (i = 2; i <= n; ++i)
		if (gt((p[i] - C).norm2(), r2))
			for (C = p[i], r2 = 0, j = 1; j < i; ++j)
				if (gt((p[j] - C).norm2(), r2))
					for (C = (p[i] + p[j]) * 0.5, r2 = (p[j] - C).norm2(),
					    k = 1;
					     k < j; ++k)
						if (gt((p[k] - C).norm2(), r2))
							C = circumCenter(p[i], p[j], p[k]),
							r2 = (p[k] - C).norm2();
	return ret ? *ret = C : 0, r2;
}
//半平面交
inline bool HPIcmp(const line u, const line v) {
	return sameDir(u, v) ? gt((fabs(u.A) + fabs(u.B)) * v.C,
	                          (fabs(v.A) + fabs(v.B)) * u.C)
	                     : u.normVec() << v.normVec();
}
inline bool geStraight(const vec2 A, const vec2 B) {
	return lt(A ^ B, 0) || le(A ^ B, 0) && lt(A * B, 0);
}
inline bool para_nega_test(const line u, const line v) {
	return parallel(u, v) && lt(u.normVec() * v.normVec(), 0)
	       && (fabs(u.A) + fabs(u.B)) * v.C + (fabs(v.A) + fabs(v.B)) * u.C
	              < -7e-14;
}
int HPI(int n, line* l, vec2* poly, vec2*& ret) {  // -1 : Unbounded, -2 :
	Infeasible int h = 0, t = -1, i, j, que[n + 5];
	std::sort(l, l + n, HPIcmp);
	n = std::unique(l, l + n, sameDir) - l;
	for (j = i = 0; i < n && j < n; ++i) {
		for (up(j, i + 1); j < n && !geStraight(l[i].normVec(), l[j].normVec());
		     ++j)
			;
		if (para_nega_test(l[i], l[j])) return -2;
	}
	if (n <= 2 || geStraight(l[n - 1].normVec(), l->normVec())) return -1;
	for (i = 0; i < n; ++i) {
		if (geStraight(l[que[t]].normVec(), l[i].normVec())) return -1;
		for (; h < t && le(l[i](poly[t - 1]), 0); --t)
			;
		for (; h < t && le(l[i](poly[h]), 0); ++h)
			;
		que[++t] = i;
		h < t ? poly[t - 1] = intersection(l[que[t - 1]], l[que[t]]) : 0;
	}
	for (; h < t && le(l[que[h]](poly[t - 1]), 0); --t)
		;
	return h == t ? -2
	              : (poly[t] = intersection(l[que[t]], l[que[h]]),
	                 ret = poly + h, t - h + 1);
}
// circles
const double pi = M_PI, pi2 = pi * 2., pi_2 = M_PI_2;
inline double angle(const vec2 u, const vec2 v) { return atan2(u ^ v, u * v); }
// intersection of circle and line
int intersection(double r2, const vec2 O, const line l, vec2* ret) {
	double d2 = dis2(O, l);
	vec2 j = l.normVec();
	if (gt(d2, r2)) return ret[0] = ret[1] = vec2(INFINITY, INFINITY), 0;
	if (ge(d2, r2)) return ret[0] = ret[1] = projection(O, l), 1;
	if (le(d2, 0)) {
		j = j * sqrt(r2 / j.norm2());
		ret[0] = O + j.trans(0, -1, 1, 0);
		ret[1] = O + j.trans(0, 1, -1, 0);
	} else {
		double T = copysign(sqrt((r2 - d2) / d2), l(O));
		j = j * (-l(O) / j.norm2());
		ret[0] = O + j.trans(1, T, -T, 1);
		ret[1] = O + j.trans(1, -T, T, 1);
	}
	return 2;
}
// area of intersection(x^2 + y^2 = r^2, triangle OBC)
double interArea(double r2, const vec2 B, const vec2 C) {
	if (eq(B ^ C, 0)) return 0;
	vec2 is[2];
	if (!intersection(r2, vec2(), line(B, C), is))
		return 0.5 * r2 * angle(B, C);
	bool b = gt(B.norm2(), r2), c = gt(C.norm2(), r2);
	if (b && c)
		return 0.5
		       * (lt(dot(*is, B, C), 0)
		              ? r2 * (angle(B, *is) + angle(is[1], C)) + (is[0] ^ is[1])
		              : r2 * angle(B, C));
	else if (b) return 0.5 * (r2 * angle(B, *is) + (*is ^ C));
	else if (c) return 0.5 * ((B ^ is[1]) + r2 * angle(is[1], C));
	else return 0.5 * (B ^ C);
}
// tangents to circle((0, 0), r) through P
int tangent(double r, const vec2 P, line* ret) {
	double Q = P.norm2() - r * r;
	if (lt(Q, 0))
		return ret[0] = ret[1] = line(INFINITY, INFINITY, INFINITY), 0;
	if (le(Q, 0)) return ret[0] = ret[1] = line(P.x, P.y, -P.norm2()), 1;
	Q = sqrt(Q) / r;
	ret[0] = line(P.x + Q * P.y, P.y - Q * P.x, -P.norm2());
	ret[1] = line(P.x - Q * P.y, P.y + Q * P.x, -P.norm2());
	return 2;
}
// tangets to circle(O, r) through P
int tangent(double r, const vec2 O, const vec2 P, line* ret) {
	int R = tangent(r, P - O, ret);
	if (R)
		ret[0].C -= ret[0].A * O.x + ret[0].B * O.y,
		    ret[1].C -= ret[1].A * O.x + ret[1].B * O.y;
	return R;
}\end{lstlisting}
\subsection{自适应辛普森}
\begin{lstlisting}
double Simpson(double L, double M, double R, double fL, double fM, double fR,
               double eps) {
	double LM = (L + M) * 0.5, fLM = f(LM), MR = (M + R) * 0.5, fMR = f(MR);
	double A = (fL + fM * 4.0 + fR) * (R - L) * sixth,
	       AL = (fL + fLM * 4.0 + fM) * (M - L) * sixth,
	       AR = (fM + fMR * 4.0 + fR) * (R - M) * sixth;
	if (fabs(AL + AR - A) < eps) return (2.0 * (AL + AR) + A) * third;
	return Simpson(L, LM, M, fL, fLM, fM, eps * 0.6)
	       + Simpson(M, MR, R, fM, fMR, fR, eps * 0.6);
}\end{lstlisting}
\clearpage\section{杂项}
Delete This
\begin{lstlisting}
\end{lstlisting}

\end{document}
