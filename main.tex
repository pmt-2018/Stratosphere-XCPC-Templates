
\documentclass[twoside]{article}
\usepackage[colorlinks,linkcolor=black]{hyperref}
\usepackage{xeCJK}
\usepackage{svg}
\usepackage{fancyhdr}
\usepackage{amsmath, amsthm}
\usepackage{listings,xcolor}
\usepackage{geometry}
\usepackage{fontspec}
\usepackage{graphicx}
\setsansfont{Monaco}
\setmonofont[Mapping={}]{Monaco}
\newcommand{\HRule}{\rule{\linewidth}{0.5mm}}
\geometry{left=2.5cm,right=2.5cm,top=2.5cm,bottom=2.5cm}
\definecolor{mygreen}{rgb}{0,0.6,0}
\definecolor{mygray}{rgb}{0.5,0.5,0.5}
\definecolor{mymauve}{rgb}{0.58,0,0.82}
\lstset{ 
  backgroundcolor=\color{white},        % choose the background color; you must add \usepackage{color} or \usepackage{xcolor}; should come as last argument
  basicstyle=\footnotesize\ttfamily,    % the size of the fonts that are used for the code
  breakatwhitespace=false,              % sets if automatic breaks should only happen at whitespace
  breaklines=true,                      % sets automatic line breaking
  captionpos=b,                         % sets the caption-position to bottom
  columns=fullflexible,
  commentstyle=\color{mygreen},         % comment style
  deletekeywords={...},                 % if you want to delete keywords from the given language
  escapeinside={\%*}{*)},               % if you want to add LaTeX within your code
  extendedchars=true,                   % lets you use non-ASCII characters; for 8-bits encodings only, does not work with UTF-8
  firstnumber=1000,                     % start line enumeration with line 1000
  frame=single,	                        % adds a frame around the code
  keepspaces=true,                      % keeps spaces in text, useful for keeping indentation of code (possibly needs columns=flexible)
  keywordstyle=\color{blue},            % keyword style
  language=c++,                         % the language of the code
  lineskip=-0.2ex,
  morekeywords={*,...},                 % if you want to add more keywords to the set
  numbers=left,                         % where to put the line-numbers; possible values are (none, left, right)
  numbersep=5pt,                        % how far the line-numbers are from the code
  numberstyle=\tiny\color{mygray},      % the style that is used for the line-numbers
  rulecolor=\color{black},              % if not set, the frame-color may be changed on line-breaks within not-black text (e.g. comments (green here))
  rulesepcolor=\color{red!20!green!20!blue!20},
  showspaces=false,                     % show spaces everywhere adding particular underscores; it overrides 'showstringspaces'
  showstringspaces=false,               % underline spaces within strings only
  showtabs=true,                        % show tabs within strings adding particular underscores
  % stepnumber=2,                       % the step between two line-numbers. If it's 1, each line will be numbered
  stringstyle=\color{mymauve}\ttfamily, % string literal style
  tabsize=4,	                        % sets default tabsize to 2 spaces
  tab={\rule[-.2\baselineskip]{.4pt}{\baselineskip}\kern 1.5em}
}
\usepackage{indentfirst}
\setlength{\parindent}{2em}
\title{Stratosphere's XCPC Templates}
\author{平流层 Stratosphere}
\pagestyle{fancy}
\fancyhf{}
\fancyhead[C]{Stratosphere's XCPC Templates, 平流层 Stratosphere}
\begin{document}\small
\begin{titlepage}
\begin{center}
\vspace*{0.5cm}\includesvg[width=0.75\textwidth]{logo.svg} \\ [2cm]
\HRule \\ [1cm]
\textbf{\Huge{Stratosphere's XCPC Templates}} \\ [0.5cm]
\HRule \\ [4cm]
\textbf{\Huge{南京大学}} \\ [1cm]
\LARGE{平流层 Stratosphere}
\vfill
\Large{\today}
\end{center}
\clearpage
\end{titlepage}
\tableofcontents\clearpage
\pagestyle{fancy}
\lfoot{}
\cfoot{\thepage}\rfoot{}
\setcounter{section}{-1}
\setcounter{page}{1}
\clearpage\section{Header}
\clearpage\section{图论}
\subsection{欧拉回路}
\begin{lstlisting}
namespace Euler {
	bool directed;
	vector<pii> G[maxn];
	vector<int> ans;
	int vis[maxm];
	int dfs(int x) {
		vector<int> t;
		while (G[x].size()) {
			auto [to, id] = G[x].back();
			G[x].pop_back();
			if (!vis[abs(id)]) {
				vis[abs(id)] = 1, t.push_back(dfs(to)), ans.push_back(id);
			}
		}
		for (int i = 1; i < t.size(); i++) {
			if (t[i] != x) ans.clear();
		}
		return t.size() ? t[0] : x;
	}
	int n, m;
	pii e[maxm];
	int deg[maxn], vv[maxn];
	void clr() {
		for (int i = 1; i <= n; i++) G[i].clear(), deg[i] = vv[i] = 0;
		for (int i = 1; i <= m; i++) vis[i] = 0;
		ans.clear();
		n = m = 0;
	}
	void addedge(int x, int y) {
		chkmax(n, x), chkmax(n, y);
		e[++m] = {x, y};
		if (directed) {
			G[x].push_back({y, m});
			++deg[x], --deg[y], vv[x] = vv[y] = 1;
		} else {
			G[x].push_back({y, m});
			G[y].push_back({x, -m});
			++deg[x], ++deg[y], vv[x] = vv[y] = 1;
		}
	}
	using vi = vector<int>;
	pair<vi, vi> work() {
		if (!m) return clr(), pair<vi, vi>{{1}, {}};
		int S = 1;
		for (int i = 1; i <= n; i++)
			if (vv[i]) S = i;
		for (int i = 1; i <= n; i++)
			if (deg[i] > 0 && deg[i] % 2 == 1) S = i;
		dfs(S);
		if ((int)ans.size() != m) return clr(), pair<vi, vi>();
		reverse(ans.begin(), ans.end());
		vi ver, edge = ans;
		if (directed) {
			ver = {e[ans[0]].fir};
			for (auto t : ans) ver.push_back(e[t].sec);
		} else {
			ver = {ans[0] > 0 ? e[ans[0]].fir : e[-ans[0]].sec};
			for (auto t : ans) ver.push_back(t > 0 ? e[t].sec : e[-t].fir);
		}
		clr();
		return {ver, edge};
	}
}  // namespace Euler\end{lstlisting}
\subsection{Tarjan-SCC}
\begin{lstlisting}
void tarjan(int u) {
	dfn[u] = low[u] = ++tim;
	in[u] = 1;
	st[++top] = u;
	for (int v : G[u]) {
		if (!dfn[v])
			tarjan(v), ckmin(low[u], low[v]);
		else if (in[v])
			ckmin(low[u], dfn[v]);
	}
	if (dfn[u] == low[u]) {
		++totc;
		int x;
		do { x = st[top--], in[x] = 0, bel[x] = totc; } while (x != u);
	}
}\end{lstlisting}
\subsection{点双}
\begin{lstlisting}
int T;  // assign = n
void tarjan(int u, int fa) {
	dfn[u] = low[u] = ++tim;
	stk[++top] = u;
	for (int v : G[u]) {
		if (v == fa) continue;
		if (!dfn[v])
			dfs(v, u), ckmin(low[u], low[v]);
		else
			ckmin(low[u], dfn[v]);
	}
	if (fa && low[u] >= dfn[fa]) {
		int y;
		++T;
		do {
			y = stk[top--];
			G2[T].push_back(y), G2[y].push_back(T);
		} while (y != u);
		G2[T].push_back(fa), G2[fa].push_back(T);
	}
}\end{lstlisting}
\subsection{边双}
\begin{lstlisting}
void tarjan(int u, int f) {
	dfn[u] = low[u] = ++tim;
	st[++top] = u;
	for (int v : G[u]) {
		if (v == f) continue;
		if (!dfn[v])
			tarjan(v, u), ckmin(low[u], low[v]);
		else
			ckmin(low[u], dfn[v]);
	}
	if (dfn[u] == low[u]) {
		++totc;
		int x;
		do { x = st[top--], in[x] = 0, bel[x] = totc; } while (x != u);
	}
}
\end{lstlisting}
\subsection{2-SAT}
构造方案时可以通过变量在图中的拓扑序确定该变量的取值。\\
如果变量$x$的拓扑序在$\lnot x$之后,那么取$x$值为真。\\
因为Tarjan 算法求强连通分量时使用了栈,所以 Tarjan 求得的 SCC
编号相当于反拓扑序。
\begin{lstlisting}
for (int i = 1; i <= n; i++)
	if (bel[i << 1] == bel[i << 1 | 1]) return puts("IMPOSSIBLE"), 0;
puts("POSSIBLE");
for (int i = 1; i <= n; i++) printf("%d ", bel[i << 1] > bel[i << 1 | 1]);\end{lstlisting}
\subsection{最大流}
Dinic 算法
\begin{lstlisting}
namespace Dinic {
	int N, S, T;
	struct edge {
		int to, nxt, cap;
	} e[maxm << 1];
	int head[maxn], cur[maxn], tot = 1;
	int d[maxn];
	void addedge(int u, int v, int c) {
		e[++tot] = (edge){v, head[u], c}, head[u] = tot;
		e[++tot] = (edge){u, head[v], 0}, head[v] = tot;
	}
	bool bfs(int S, int T) {
		queue<int> q;
		for (int i = 1; i <= N; i++) d[i] = 0;
		d[S] = 1;
		q.push(S);
		while (!q.empty()) {
			int u = q.front();
			q.pop();
			for (int i = head[u]; i; i = e[i].nxt) {
				int v = e[i].to;
				if (e[i].cap && !d[v]) {
					d[v] = d[u] + 1, q.push(v);
					if (v == T) return true;
				}
			}
		}
		return false;
	}
	int dfs(int u, int f) {
		if (u == T) return f;
		int r = f;
		for (int& i = cur[u]; i && r; i = e[i].nxt) {
			int v = e[i].to;
			if (e[i].cap && d[v] == d[u] + 1) {
				int x = dfs(v, min(e[i].cap, r));
				if (!x) d[v] = 0;
				e[i].cap -= x, e[i ^ 1].cap += x;
				r -= x;
			}
		}
		return f - r;
	}
	ll work(int _N, int _S, int _T) {
		N = _N, S = _S, T = _T;
		ll ans = 0;
		while (bfs(S, T)) {
			for (int i = 1; i <= N; i++) cur[i] = head[i];

			ans += 1ll * dfs(S, INF);
		}
		return ans;
	}
}  // namespace Dinic
\end{lstlisting}
ISAP 算法
\begin{lstlisting}
namespace ISAP {
	int N, S, T;
	struct edge {
		int to, nxt, cap;
	} e[maxm << 1];
	int head[maxn], cur[maxn], gap[maxn], dis[maxn], tot = 1;
	void addedge(int u, int v, int w) {
		e[++tot] = {v, head[u], w}, head[u] = tot;
		e[++tot] = {u, head[v], 0}, head[v] = tot;
	}
	int ISAP(int u, int lim) {
		if (u == T) return lim;
		int res = 0;
		for (int& i = cur[u]; i; i = e[i].nxt) {
			int v = e[i].to;
			if (e[i].cap && dis[u] == dis[v] + 1) {
				ll det = ISAP(v, min(lim, e[i].cap));
				e[i].cap -= det, e[i ^ 1].cap += det;
				lim -= det, res += det;
				if (!lim) return res;
			}
		}
		cur[u] = head[u];
		if (!--gap[dis[u]]) dis[S] = N + 1;
		gap[++dis[u]]++;
		return res;
	}
	ll work(int _N, int _S, int _T) {
		S = _S, T = _T, N = _N;
		ll res = 0;
		while (dis[S] <= N) res += 1ll * ISAP(S, INF);
		return res;
	}
}  // namespace ISAP
\end{lstlisting}
HLPP 算法
\begin{lstlisting}
namespace HLPP {  // by ProjectEMmm
	int N, S, T;
	struct edge {
		int to, nxt, cap;
	} e[maxm << 1];
	int head[maxn], tot = 1;

	int d[maxn], num[maxn];
	stack<int> lib[maxn];
	ll ex[maxn];
	int level = 0;
	void addedge(int u, int v, int c) {
		e[++tot] = {v, head[u], c}, head[u] = tot;
		e[++tot] = {u, head[v], 0}, head[v] = tot;
	}
	int Push(int u) {
		bool init = (u == S);
		for (int i = head[u]; i; i = e[i].nxt) {
			const int &v = e[i].to, &c = e[i].cap;
			if (!c || init == false && d[u] != d[v] + 1) continue;
			ll k = init ? c : min((ll)c, ex[u]);
			if (v != S && v != T && !ex[v] && d[v] < INF)
				lib[d[v]].push(v), level = max(level, d[v]);
			ex[u] -= k, ex[v] += k, e[i].cap -= k, e[i ^ 1].cap += k;
			if (!ex[u]) return 0;
		}
		return 1;
	}
	void Relabel(int x) {
		d[x] = INF;
		for (int i = head[x]; i; i = e[i].nxt)
			if (e[i].cap) d[x] = min(d[x], d[e[i].to]);
		if (++d[x] < N) {
			lib[d[x]].push(x);
			level = max(level, d[x]);
			++num[d[x]];
		}
	}
	bool BFS() {
		for (int i = 1; i <= N; ++i) {
			d[i] = INF;
			num[i] = 0;
		}
		queue<int> q;
		q.push(T), d[T] = 0;
		while (!q.empty()) {
			int u = q.front();
			q.pop();
			num[d[u]]++;
			for (int i = head[u]; i; i = e[i].nxt) {
				const int& v = e[i].to;
				if (e[i ^ 1].cap && d[v] > d[u] + 1) d[v] = d[u] + 1, q.push(v);
			}
		}
		return d[S] != INF;
	}
	int Select() {
		while (lib[level].size() == 0 && level > -1) level--;
		return level == -1 ? 0 : lib[level].top();
	}
	ll work(int _N, int _S, int _T) {
		N = _N, S = _S, T = _T;
		if (!BFS()) return 0;
		d[S] = N;
		Push(S);
		int x;
		while (x = Select()) {
			lib[level].pop();
			if (!Push(x)) continue;
			if (!--num[d[x]])
				for (int i = 1; i <= N; ++i)
					if (i != S && i != T && d[i] > d[x] && d[i] < N + 1)
						d[i] = N + 1;
			Relabel(x);
		}
		return ex[T];
	}
}  // namespace HLPP\end{lstlisting}
\subsection{最小费用最大流}
\begin{lstlisting}
namespace MCMF {
	using pr = pair<ll, int>;
	int N, S, T;
	struct edge {
		int to, nxt, cap, w;
	} e[maxm << 1];
	int head[maxn], tot = 1;
	void addedge(int x, int y, int cap, int w) {
		e[++tot] = {y, head[x], cap, w}, head[x] = tot;
		e[++tot] = {x, head[y], 0, -w}, head[y] = tot;
	}
	ll d[maxn], dis[maxn];
	int vis[maxn], fr[maxn];
	bool spfa() {
		queue<int> Q;
		fill(d + 1, d + N + 1, 1e18);  // CHECK
		for (d[S] = 0, Q.push(S); !Q.empty();) {
			int x = Q.front();
			Q.pop();
			vis[x] = 0;
			for (int i = head[x]; i; i = e[i].nxt)
				if (e[i].cap && d[e[i].to] > d[x] + e[i].w) {
					d[e[i].to] = d[x] + e[i].w;
					fr[e[i].to] = i;
					if (!vis[e[i].to]) vis[e[i].to] = 1, Q.push(e[i].to);
				}
		}
		return d[T] < 1e17;  // 如果只是最小费用流,当d < 0继续增广
	}
	bool dijkstra() {  // 正常题目不需要 dijk
		priority_queue<pr, vector<pr>, greater<pr>> Q;
		for (int i = 1; i <= N; ++i)
			dis[i] = d[i], d[i] = 1e18, vis[i] = fr[i] = 0;  // CHECK
		Q.emplace(d[S] = 0, S);
		while (!Q.empty()) {
			int x = Q.top().second;
			Q.pop();
			if (vis[x]) continue;
			vis[x] = 1;
			for (int i = head[x]; i; i = e[i].nxt) {
				const ll v = e[i].w + dis[x] - dis[e[i].to];
				if (e[i].cap && d[e[i].to] > d[x] + v) {
					fr[e[i].to] = i;
					Q.emplace(d[e[i].to] = d[x] + v, e[i].to);
				}
			}
		}
		for (int i = 1; i <= N; ++i) d[i] += dis[i];  // CHECK
		return d[T] < 1e17;
	}
	std::pair<ll, ll> work(int _N, int _S, int _T) {
		N = _N, S = _S, T = _T;
		spfa();  // 如果初始有负权且要 dijk
		ll f = 0, c = 0;
		for (; dijkstra();) {  // 正常可以用 spfa
			ll fl = 1e18;
			for (int i = fr[T]; i; i = fr[e[i ^ 1].to])
				fl = min((ll)e[i].cap, fl);
			for (int i = fr[T]; i; i = fr[e[i ^ 1].to])
				e[i].cap -= fl, e[i ^ 1].cap += fl;
			f += fl, c += fl * d[T];
		}
		return make_pair(f, c);
	}
}  // namespace MCMF\end{lstlisting}
\subsection{匹配}
\subsubsection{二分图最大匹配-Hungary}
\begin{lstlisting}
// 匈牙利,左到右单向边, O (M |match|)
int vis[maxn], match[maxn];
bool dfs(int u) {
	for (int v : G[u]) {
		if (vis[v]) continue;
		vis[v] = 1;
		if (!match[v] || dfs(match[v])) return match[v] = u, 1;
	}
	return 0;
}
int work() {
	for (int i = 1; i <= nl; i++)
		if (dfs(i)) fill(vis + 1, vis + nr + 1, 0);
}
// 匈牙利,左到右单向边,bitset, O (n^2|match|/w)
bitset<N> G[N], unvis;
int match[N];
bool dfs(int u) {
	for (auto s = G[u];;) {
		s &= unvis;
		int v = s._Find_first();
		if (v == N) return 0;
		unvis.reset(v);
		if (!match[v] || dfs(match[v])) return match[v] = u, 1;
	}
	return 0;
}
int work() {
	unvis.set();
	for (int i = 1; i <= nl; i++)
		if (dfs(i)) unvis.set();
}\end{lstlisting}
\subsubsection{二分图最大匹配-HK}
\begin{lstlisting}
// HK, 左到右单向边, O(M \sqrt{|match|})
int matchl[maxn], matchr[maxn], a[maxn], p[maxn];
int HK() {
	while (true) {
		for (int i = 1; i <= nl; i++) a[i] = p[i] = 0;
		queue<int> Q;
		while (!Q.empty()) Q.pop();
		for (int i = 1; i <= nl; i++)
			if (!matchl[i]) a[i] = p[i] = i, Q.push(i);
		int succ = 0;
		while (!Q.empty()) {
			int u = Q.front();
			Q.pop();
			if (matchl[a[u]]) continue;
			for (int v : G[u]) {
				if (!matchr[v]) {
					for (succ = 1; v; u = p[u])
						matchr[v] = u, swap(matchl[u], v);
					break;
				}
				if (!p[matchr[v]])
					Q.push(matchr[v]), p[matchr[v]] = u, a[matchr[v]] = a[u];
			}
		}
		if (!succ) break;
	}
}\end{lstlisting}
\subsubsection{二分图最大权匹配-KM}
\begin{lstlisting}
// KM 二分图最大权匹配 复杂度O(n^3)
namespace KM {
	int nl, nr;
	ll e[maxn][maxn], lw[maxn], rw[maxn], mnw[maxn];
	int lpr[maxn], rpr[maxn], vis[maxn], fa[maxn];
	void addedge(int x, int y, ll w) {
		ckmax(e[x][y], w), ckmax(lw[x], w);
	}
	void work(int x) {
		int xx = x;
		for (int i = 1; i <= nr; i++) vis[i] = 0, mnw[i] = 1e18;
		while (true) {
			for (int i = 1; i <= nr; i++)
				if (!vis[i] && mnw[i] >= lw[x] + rw[i] - e[x][i])
					ckmin(mnw[i], lw[x] + rw[i] - e[x][i]), fa[i] = x;
			ll mn = 1e18;
			int y = -1;
			for (int i = 1; i <= nr; i++)
				if (!vis[i] && mn >= mnw[i]) ckmin(mn, mnw[i]), y = i;
			lw[xx] -= mn;
			for (int i = 1; i <= nr; i++)
				if (vis[i])
					rw[i] += mn, lw[rpr[i]] -= mn;
				else
					mnw[i] -= mn;
			if (rpr[y])
				x = rpr[y], vis[y] = 1;
			else {
				while (y) rpr[y] = fa[y], swap(y, lpr[fa[y]]);
				return;
			}
		}
	}
	void init(int _nl, int _nr) {
		nl = _nl, nr = _nr;
		if (nl > nr) nr = nl;
		for (int i = 1; i <= nl; i++) lw[i] = -1e18;
		for (int i = 1; i <= nl; i++)
			for (int j = 1; j <= nr; j++) e[i][j] = 0;  // or -1e18
	}
	ll work() {
		for (int i = 1; i <= nl; i++) work(i);
		ll tot = 0;
		for (int i = 1; i <= nl; i++) tot += e[i][lpr[i]];
		return tot;
	}
}  // namespace KM\end{lstlisting}
\subsubsection{一般图最大匹配-带花树}
\begin{lstlisting}
namespace blossom {
	vector<int> G[maxn];
	int f[maxn];
	int n, match[maxn];
	int getfa(int x) {
		return f[x] == x ? x : f[x] = getfa(f[x]);
	}
	void addedge(int x, int y) {
		G[x].push_back(y), G[y].push_back(x);
	}
	int pre[maxn], mk[maxn];
	int vis[maxn], T;
	queue<int> q;
	int LCA(int x, int y) {
		T++;
		for (;; x = pre[match[x]], swap(x, y))
			if (vis[x = getfa(x)] == T)
				return x;
			else
				vis[x] = x ? T : 0;
	}
	void flower(int x, int y, int z) {
		while (getfa(x) != z) {
			pre[x] = y;
			y = match[x];
			f[x] = f[y] = z;
			x = pre[y];
			if (mk[y] == 2) q.push(y), mk[y] = 1;
		}
	}
	void aug(int s) {
		for (int i = 1; i <= n; i++) pre[i] = mk[i] = vis[i] = 0, f[i] = i;
		q = {};
		mk[s] = 1;
		q.push(s);
		while (q.size()) {
			int x = q.front();
			q.pop();
			for (int v : G[x]) {
				int y = v, z;
				if (mk[y] == 2) continue;
				if (mk[y] == 1)
					z = LCA(x, y), flower(x, y, z), flower(y, x, z);
				else if (!match[y]) {
					for (pre[y] = x; y;)
						x = pre[y], match[y] = x, swap(y, match[x]);
					return;
				} else
					pre[y] = x, mk[y] = 2, q.push(match[y]), mk[match[y]] = 1;
			}
		}
	}
	int work() {
		for (int i = 1; i <= n; i++) if (!match[i]) aug(i);
		int res = 0;
		for (int i = 1; i <= n; i++) res += match[i] > i;
		return res;
	}
}  // namespace blossom\end{lstlisting}
\subsubsection{一般图最大权匹配}
待补充\subsection{最短路相关}
\subsubsection{差分约束}
$x_i$ 向 $x_j$ 连一条权值为 $c$ 的有向边表示 $x_j - x_i \le c$。

用 BF 判断是否存在负环,存在即无解。\subsubsection{最小环}
记原图中 $u,v$ 之间边的边权为 $val\left(u,v\right)$。

我们注意到 Floyd 算法有一个性质:在最外层循环到点 $k$ 时(尚未开始第 $k$ 次循环),最短路数组 $dis$ 中,$dis_{u,v}$ 表示的是从 $u$ 到 $v$ 且仅经过编号在 $\left[1, k\right)$ 区间中的点的最短路。

由最小环的定义可知其至少有三个顶点,设其中编号最大的顶点为 $w$,环上与 $w$ 相邻两侧的两个点为 $u,v$,则在最外层循环枚举到 $k=w$ 时,该环的长度即为 $dis_{u,v}+val\left(v,w\right)+val\left(w,u\right)$。

故在循环时对于每个 $k$ 枚举满足 $i<k,j<k$ 的 $(i,j)$,更新答案即可。\subsubsection{Steiner树}
状态设计:$dp(i,S)$ 以 $i$ 为根,树中关键点集合为 $S$ 的最小值。

\begin{enumerate}
    \item[1.] 树根度数不为 $1$ ,考虑拆分成两个子集 $T,S-T$:
    $$ dp(i,S) \gets dp(i,S-T)+dp(i,T) $$
    \item[2.] 树根度数为 $1$ :
    $$ dp(i,S) \gets dp(j,S)+w(i,j) $$

    相当于超级源到每个顶点距离为$dp(i,S)$,求到每个顶点的最短路,dij即可。
\end{enumerate}
\subsection{流和匹配的建模技巧}
\subsubsection{二分图相关}
\begin{itemize}
    \item 二分图最小点覆盖:等于最大匹配$|match|$。从每一个非匹配点出发,沿着非匹配边正向进行遍历,沿着匹配边反向进行遍历到的点进行标记。选取左部点中没有被标记过的点,右部点中被标记过的点,则这些点可以形成该二分图的最小点覆盖。
    \item 二分图最大独立集:等于$n-|match|$,考虑最小点覆盖给所有边都至少有一边有点,取反后必然为最大独立集。
    \item 二分图最小边覆盖:等于$n-|match|$,考虑最坏情况每个顶点都要一条边,一个匹配能减小1的贡献。
    \item 最大团:等于补图的最大独立集。
    \item 最小路径覆盖:对于每条有向边$(u,v)$,拆成$u\to v+n$,$u$为进入$u$,$v+n$为从v离开,则答案为$n-|match|$。
    \item Hall Theorem: 对于左部顶点集$X$,$\forall S\subseteq X, |N(S)|\ge |S|\iff$ 存在完美匹配。
\end{itemize}

\subsubsection{网络流相关}

\begin{itemize}
    \item 二分图最大权独立集:考虑连边$(S, x, w_x)$,原图边$(x,y,\infty)$,$(y,T,w_y)$,变为最小割。
    \item 最大权闭合子图:正权$w_u$连$(S,u,w_u)$,负权$w_v$连$(v,T,-w_v)$,原图边连$\infty$。此时最小割之后源点$S$能到达的点即为最大权闭合子图,答案即为正权和$-\text{mincut}$。
    \item 无源汇上下界可行流:建源汇$S,T$,$l(u,v),r(u,v)$分别为流量上下界。记$d(i) = \sum l(u,i) - \sum l(i,v)$。
    \begin{itemize}
        \item 原边$(u,v)$连$(u,v,r(u,v)-l(u,v))$。
        \item 对于每个点$u$,若$d_u>0$,连$(S, u, d_u)$。
        \item 若$d_u<0$,连$(u, T, -d_u)$。
    \end{itemize}
    若$S$的出边全部流满则存在解。
    \item 有源汇上下界可行流:原图源汇连边$(T\to S,(0,\infty))$,则转化为无源汇。
    \item 有源汇上下界最大流:从 $T$ 到 $S$ 连一条下界为 $0$,上界为 $+\infty$ 的边,转化为无源汇网络。按照无源汇上下界可行流的做法求一次无源汇上下界超级源$SS$到超级汇$TT$的最大流。
    删去所有附加边,在上一步的\textbf{残量网络}基础上,求一次 $S$ 到 $T$ 的最大流。两者之和即为答案。
    \item 有源汇上下界最小流:从 $T$ 到 $S$ 连一条下界为 $0$,上界为 $+\infty$ 的边,转化为无源汇网络。按照无源汇上下界可行流的做法求一次无源汇上下界超级源$SS$到超级汇$TT$的最大流。
    删去所有附加边,在上一步的\textbf{残量网络}基础上,求一次 $T$ 到 $S$ 的最大流。两者之差即为答案。
    \item 最小费用可行流:同有源汇上下界可行流,在超级源汇跑最小费用最大流,答案为费用+下界流量的费用。
    \item 平面图最小割 = 对偶图最短路
\end{itemize}
\subsection{三四元环计数}
\begin{lstlisting}
static int id[maxn], rnk[maxn];
for (int i = 1; i <= n; i++) id[i] = i;
sort(id + 1, id + n + 1, [](int x, int y) {
	return pii{deg[x], x} < pii{deg[y], y};
});
for (int i = 1; i <= n; i++) rnk[id[i]] = i;
for (int i = 1; i <= n; i++)
	for (int v : G[i])
		if (rnk[v] > rnk[i]) G2[i].push_back(v);
int ans3 = 0;  // 3-cycle
for (int i = 1; i <= n; i++) {
	static int vis[maxn];
	for (int v : G2[i]) vis[v] = 1;
	for (int v1 : G2[i])
		for (int v2 : G2[v1])
			if (vis[v2]) ++ans3;  // (i,v1,v2)
	for (int v : G2[i]) vis[v] = 0;
}
ll ans4 = 0;  // 4-cycle
for (int i = 1; i <= n; i++) {
	static int vis[maxn];
	for (int v1 : G[i])
		for (int v2 : G2[v1])
			if (rnk[v2] > rnk[i]) ans4 += vis[v2], vis[v2]++;
	for (int v1 : G[i])
		for (int v2 : G2[v1]) vis[v2] = 0;
}\end{lstlisting}
\subsection{支配树}
\begin{lstlisting}
namespace Dom_DAG {
	int idom[maxn];
	vector<int> G[maxn], ANS[maxn];  // ANS: final tree
	int deg[maxn];
	int fa[maxn][25], dep[maxn];
	int lca(int x, int y) {
		if (dep[x] < dep[y]) swap(x, y);
		for (int i = 20; i >= 0; i--)
			if (fa[x][i] && dep[fa[x][i]] >= dep[y]) x = fa[x][i];
		if (x == y) return x;
		for (int i = 20; i >= 0; i--)
			if (fa[x][i] != fa[y][i]) x = fa[x][i], y = fa[y][i];
		return fa[x][0];
	}
	void work() {
		queue<int> q;
		q.push(1);
		while (!q.empty()) {
			int x = q.front();
			q.pop();
			ANS[idom[x]].push_back(x);
			fa[x][0] = idom[x];
			dep[x] = dep[idom[x]] + 1;
			for (int i = 1; i <= 20; i++) fa[x][i] = fa[fa[x][i - 1]][i - 1];
			for (int v : G[x]) {
				--deg[v];
				if (!deg[v]) q.push(v);
				if (!idom[v])
					idom[v] = x;
				else
					idom[v] = lca(idom[v], x);
			}
		}
	}
}  // namespace Dom_DAG
namespace Dom {
	vector<int> G[maxn], rG[maxn];
	int dfn[maxn], id[maxn], anc[maxn], cnt;
	void dfs(int x) {
		id[dfn[x] = ++cnt] = x;
		for (int v : G[x])
			if (!dfn[v]) {
				Dom_DAG::G[x].push_back(v);
				Dom_DAG::deg[v]++;
				anc[v] = x;
				dfs(v);
			}
	}
	int fa[maxn], mn[maxn];
	int find(int x) {
		if (x == fa[x]) return x;
		int tmp = fa[x];
		fa[x] = find(fa[x]);
		ckmin(mn[x], mn[tmp]);
		return fa[x];
	}
	int semi[maxn];
	void work() {
		dfs(1);
		for (int i = 1; i <= n; i++) fa[i] = i, mn[i] = 1e9, semi[i] = i;
		for (int w = n; w >= 2; w--) {
			int x = id[w];
			int cur = 1e9;
			if (w > cnt) continue;
			for (int v : rG[x]) {
				if (!dfn[v]) continue;
				if (dfn[v] < dfn[x])
					ckmin(cur, dfn[v]);
				else
					find(v), ckmin(cur, mn[v]);
			}
			semi[x] = id[cur];
			mn[x] = cur;
			fa[x] = anc[x];
			Dom_DAG::G[semi[x]].push_back(x);
			Dom_DAG::deg[x]++;
		}
	}
	void addedge(int x, int y) {
		G[x].push_back(y), rG[y].push_back(x);
	}
}  // namespace Dom\end{lstlisting}
\subsection{图论计数}
\subsubsection{Prufer序列}
有标号无根树和其prufer编码一一对应, 一颗 $n$ 个点的树, 其prufer编码长度为 $n - 2$, 且度数为 $d_i$ 的点在prufer编码中出现 $di - 1$ 次.

由树得到序列: 总共需要 $n - 2$ 步, 第 $i$ 步在当前的树中寻找具有最小标号的叶子节点, 将与其相连的点的标
号设为Prufer序列的第 $i$ 个元素 $p_i$ , 并将此叶子节点从树中删除, 直到最后得到一个长度为 $n - 2$ 的Prufer 序
列和一个只有两个节点的树.

由序列得到树: 先将所有点的度赋初值为 $1$, 然后加上它的编号在Prufer序列中出现的次数, 得到每个点的度;
执行 $n - 2$ 步, 第 $i$ 步选取具有最小标号的度为 $1$ 的点 $u$ 与 $v = p_i$ 相连, 得到树中的一条边, 并将 $u$ 和 $v$ 的度
减一. 最后再把剩下的两个度为 $1$ 的点连边, 加入到树中.

推论:

\begin{itemize}
    \item $n$ 个点完全图, 要求每个点度数依次为 $d_1, d_2 ,\cdots,d_n$, 这样生成树的棵树为: $\dfrac{(n-2)!}{\prod (d_i-1)!}$
    \item 左边有 $n_1$ 个点, 右边有 $n_2$ 个点的完全二分图的生成树棵树为 $n_1^{n_2-1}\times n_2^{n_1-1}$
    \item $m$个连通块, 每个连通块有 $c_i$ 个点, 把他们全部连通的生成树方案数: $(\sum c_i)^{m-2} \prod c_i$
\end{itemize}

\subsubsection{无标号树计数}
\begin{enumerate}
    \item[(1)]有根树计数: $$f_n = \dfrac{ \sum_{i=1}^{n-1} f_{n-i} \sum_{d \mid i} f_d \cdot d}{n-1}$$
    
    记$g_i = \sum_{d \mid i} f_d \cdot d$ 即可做到$\Theta(n^2)$。
    \item[(2)]无根树计数:
    
    当 $n$ 是奇数时

    如果根不是重心,必然存在恰好一个子树,它的大小超过 $\left\lfloor\dfrac n2\right\rfloor$(设它的大小为 $k$)减去这种情况即可。
    
    因此答案为
    $$f_n-\sum_{k=\lfloor\frac n2\rfloor+1}^{n-1}f_k\cdot f_{n-k}$$
    
    当 $n$ 是偶数时
    
    有可能存在两个重心,且其中一个是根(即存在一棵子树大小恰为 $\dfrac n2$),额外减去$\dbinom{f_{\frac n2}}2$即可
\end{enumerate}

\subsubsection{有标号DAG计数}

$$
F_i=\sum_{j=1}^i \binom{i}{j}(-1)^{j+1}2^{j(i-j)}F_{i-j}
$$

想法是按照拓扑序分层,每次剥开所有入度为零的点。

\subsubsection{有标号连通简单图计数}

记$g(n)=2^{\binom n2}$为有标号简单图数量,$c(n)$为有标号简单连通图数量,那么枚举$1$所在连通块大小,有

$$
g(n) = \sum_{i=1}^n \dbinom{n-1}{i-1}c(i)g(n-i)
$$

易递推求$c(n)$。多项式做法考虑exp组合意义即可。

\subsubsection{生成树计数}
Kirchhoff Matrix $T = Deg - A$, $Deg$ 是度数对角阵, $A$ 是邻接矩阵. 

无向图度数矩阵是每个点度数; 有向图度数矩阵是每个点入度.

邻接矩阵 $A[u][v]$ 表示 $u \to v$ 边个数, 重边按照边数计算, 自环不计入度数.

无向图生成树计数: $c = |K$的任意$1$个 $n-1$ 阶主子式$|$

有向图外向树计数: $c = |$去掉根所在的那阶得到的主子式$|$

若求边权和则邻接矩阵可以设为$(1+wx)$,相当于一次项的系数。

\subsubsection{BEST定理}
设 $G$ 是有向欧拉图,$k$ 为任意顶点,那么 $G$ 的不同欧拉回路总数 $\mathrm{ec}(G)$ 是

$$
\mathrm{ec}(G) = t^\mathrm{root}(k)\prod_{v\in V}(\deg (v) - 1)!.
$$

$t^\mathrm{root}(k)$为以$k$为根的外向树个数。\clearpage\section{树论}
\subsection{快速LCA}
查询$[dfn_u + 1 , dfn_v]$深度最小节点的父亲\\
可以简化为在 ST 表的最底层记录父亲,比较时取时间戳较小的结点。\\
取决于st表实现可以做到$O(n)$ or $O(n\log n)$
\begin{lstlisting}
int getmin(int x, int y) {
	return dfn[x] < dfn[y] ? x : y;
}

void dfs(int u, int f) {
	dfn[u] = ++tim;
	a[dfn[u]] = f;
	for (int v : e[u])
		if (v != f) dfs(v, u);
}
int lca(int u, int v) {
	if (u == v) return u;
	if ((u = dfn[u]) > (v = dfn[v])) swap(u, v);
	return RMQ(dfn[u] + 1, dfn[v]);
}\end{lstlisting}
\subsection{虚树}
\begin{lstlisting}

\end{lstlisting}
\subsection{长链剖分}
\begin{lstlisting}

\end{lstlisting}
\subsection{静态点分治}
\begin{lstlisting}

\end{lstlisting}
\subsection{点分树}
\begin{lstlisting}

\end{lstlisting}
\subsection{动态dp}
\begin{lstlisting}

\end{lstlisting}
\subsection{树上背包}
\begin{lstlisting}

\end{lstlisting}
\clearpage\section{数论}
Delete This
\begin{lstlisting}
\end{lstlisting}
\clearpage\section{数学}
Delete This
\begin{lstlisting}
\end{lstlisting}
\clearpage\section{字符串}
Delete This
\begin{lstlisting}
\end{lstlisting}
\clearpage\section{数据结构}
Delete This
\begin{lstlisting}
\end{lstlisting}
\clearpage\section{计算几何}
Delete This
\begin{lstlisting}
\end{lstlisting}
\clearpage\section{杂项}
Delete This
\begin{lstlisting}
\end{lstlisting}

\end{document}
