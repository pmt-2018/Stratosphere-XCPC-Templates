\subsubsection{Prufer序列}
有标号无根树和其prufer编码一一对应, 一颗 $n$ 个点的树, 其prufer编码长度为 $n - 2$, 且度数为 $d_i$ 的点在prufer编码中出现 $di - 1$ 次.

由树得到序列: 总共需要 $n - 2$ 步, 第 $i$ 步在当前的树中寻找具有最小标号的叶子节点, 将与其相连的点的标
号设为Prufer序列的第 $i$ 个元素 $p_i$ , 并将此叶子节点从树中删除, 直到最后得到一个长度为 $n - 2$ 的Prufer 序
列和一个只有两个节点的树.

由序列得到树: 先将所有点的度赋初值为 $1$, 然后加上它的编号在Prufer序列中出现的次数, 得到每个点的度;
执行 $n - 2$ 步, 第 $i$ 步选取具有最小标号的度为 $1$ 的点 $u$ 与 $v = p_i$ 相连, 得到树中的一条边, 并将 $u$ 和 $v$ 的度
减一. 最后再把剩下的两个度为 $1$ 的点连边, 加入到树中.

推论:

\begin{itemize}
    \item $n$ 个点完全图, 要求每个点度数依次为 $d_1, d_2 ,\cdots,d_n$, 这样生成树的棵树为: $\dfrac{(n-2)!}{\prod (d_i-1)!}$
    \item 左边有 $n_1$ 个点, 右边有 $n_2$ 个点的完全二分图的生成树棵树为 $n_1^{n_2-1}\times n_2^{n_1-1}$
    \item $m$个连通块, 每个连通块有 $c_i$ 个点, 把他们全部连通的生成树方案数: $(\sum c_i)^{m-2} \prod c_i$
\end{itemize}

\subsubsection{无标号树计数}
\begin{enumerate}
    \item[(1)]有根树计数: $$f_n = \dfrac{ \sum_{i=1}^{n-1} f_{n-i} \sum_{d \mid i} f_d \cdot d}{n-1}$$
    
    记$g_i = \sum_{d \mid i} f_d \cdot d$ 即可做到$\Theta(n^2)$。
    \item[(2)]无根树计数:
    
    当 $n$ 是奇数时

    如果根不是重心,必然存在恰好一个子树,它的大小超过 $\left\lfloor\dfrac n2\right\rfloor$(设它的大小为 $k$)减去这种情况即可。
    
    因此答案为
    $$f_n-\sum_{k=\lfloor\frac n2\rfloor+1}^{n-1}f_k\cdot f_{n-k}$$
    
    当 $n$ 是偶数时
    
    有可能存在两个重心,且其中一个是根(即存在一棵子树大小恰为 $\dfrac n2$),额外减去$\dbinom{f_{\frac n2}}2$即可
\end{enumerate}

\subsubsection{有标号DAG计数}

$$
F_i=\sum_{j=1}^i \binom{i}{j}(-1)^{j+1}2^{j(i-j)}F_{i-j}
$$

想法是按照拓扑序分层,每次剥开所有入度为零的点。

\subsubsection{有标号连通简单图计数}

记$g(n)=2^{\binom n2}$为有标号简单图数量,$c(n)$为有标号简单连通图数量,那么枚举$1$所在连通块大小,有

$$
g(n) = \sum_{i=1}^n \dbinom{n-1}{i-1}c(i)g(n-i)
$$

易递推求$c(n)$。多项式做法考虑exp组合意义即可。

\subsubsection{生成树计数}
Kirchhoff Matrix $T = Deg - A$, $Deg$ 是度数对角阵, $A$ 是邻接矩阵. 

无向图度数矩阵是每个点度数; 有向图度数矩阵是每个点入度.

邻接矩阵 $A[u][v]$ 表示 $u \to v$ 边个数, 重边按照边数计算, 自环不计入度数.

无向图生成树计数: $c = |K的任意1个 n-1 阶主子式|$

有向图外向树计数: $c = |去掉根所在的那阶得到的主子式|$

若求边权和则邻接矩阵可以设为$(1+wx)$,相当于一次项的系数。

\subsubsection{BEST定理}
设 $G$ 是有向欧拉图,$k$ 为任意顶点,那么 $G$ 的不同欧拉回路总数 $\mathrm{ec}(G)$ 是

$$
\mathrm{ec}(G) = t^\mathrm{root}(G,k)\prod_{v\in V}(\deg (v) - 1)!.
$$



